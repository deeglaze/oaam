\documentclass[preprint]{sigplanconf}
\usepackage{alltt}
\usepackage{amsmath}
\usepackage{amssymb}
\usepackage{stmaryrd}
% \usepackage{xltxtra}
% \setmonofont[Scale=MatchLowercase]{DejaVu Sans Mono}
%
\newcommand{\superscript}[1]{\ensuremath{^{#1}}}
\newcommand{\subscript}[1]{\ensuremath{_{#1}}}
\newcommand{\tuple}[3][\ ]{{\tt #2{#1}}({#3})}

% Values
\newcommand{\clos}[1]{\tuple{clos}{#1}}
\newcommand{\rlos}[1]{\tuple{rlos}{#1}}

% States
\newcommand{\ev}[2][\ ]{\tuple[#1]{ev}{#2}}
\newcommand{\co}[1]{\tuple{co}{#1}}
\newcommand{\ap}[2][\ ]{\tuple[#1]{ap}{#2}}
\newcommand{\ans}[1]{\tuple{ans}{#1}}

% Continuations
\newcommand{\kmt}{\tt mt}
\newcommand{\kar}[2][\ ]{\tuple[#1]{ar}{#2}}
\newcommand{\kfn}[2][\ ]{\tuple[#1]{fn}{#2}}
\newcommand{\kif}[2][\ ]{\tuple[#1]{fi}{#2}}
\newcommand{\kuop}[2][\ ]{\tuple[#1]{oa}{#2}}
\newcommand{\kbopa}[2][\ ]{\tuple[#1]{oa1}{#2}}
\newcommand{\kbopb}[2][\ ]{\tuple[#1]{oa2}{#2}}

% Implementation forms
\newcommand{\generator}{{\tt generator}}
\newcommand{\yield}[1]{{\tt yield} #1}

% Syntax
\newcommand{\syntax}[1]{{\tt #1}}
\newcommand{\sapp}[3][\ ]{\tuple[#1]{app}{#2,#3}}
\newcommand{\slam}[3][\ ]{\tuple[#1]{lam}{#2,#3}}
\newcommand{\srec}[4][\ ]{\tuple[#1]{rec}{#2,#3,#4}}
\newcommand{\svar}[2][\ ]{\tuple[#1]{var}{#2}}
\newcommand{\snum}[2][\ ]{\tuple[#1]{num}{#2}}
\newcommand{\sbln}[2][\ ]{\tuple[#1]{bool}{#2}}
\newcommand{\sif}[4][\ ]{\tuple[#1]{if}{#2,#3,#4}}
\newcommand{\sop}[2][\ ]{\tuple[#1]{op}{#2}}
\newcommand{\sopu}[3][\ ]{\tuple[#1]{op}{#2,#3}}
\newcommand{\sopb}[4][\ ]{\tuple[#1]{op2}{#2,#3,#4}}
\newcommand{\strue}{{\tt tt}}
\newcommand{\sfalse}{{\tt ff}}
\newcommand{\saddone}{{\tt add1}}
\newcommand{\ssubone}{{\tt sub1}}
\newcommand{\szerohuh}{\syntax{zero?}}
\newcommand{\szero}{\syntax{0}}
\newcommand{\slit}[2][\ ]{\tuple[#1]{lit}{#2}}

\newcommand{\sNum}{\syntax{Z}}

% Metavariables
\newcommand{\maddr}{a}
\newcommand{\mvar}{x}
\newcommand{\mvarf}{f}
\newcommand{\mexp}{e}
\newcommand{\mexpi}[1]{e_{#1}}
\newcommand{\mexpf}{f}
\newcommand{\menv}{\rho}
\newcommand{\mkont}{\kappa}
\newcommand{\msto}{\sigma}
\newcommand{\mop}{o}
\newcommand{\mval}{v}
\newcommand{\mnum}{z}
\newcommand{\mbln}{b}
\newcommand{\mvalx}[1]{#1}
\newcommand\machstep{\longmapsto}
\newcommand\multimachstep{\longmapsto\!\!\!\!\!\rightarrow}
\newcommand{\mlit}{l}
\newcommand{\mstate}{\varsigma}
\newcommand{\mcomp}{k}
\newcommand{\mcompi}[1]{\mcomp_{#1}}
\newcommand{\interpdelta}{\Delta}

\newcommand{\compile}[1]{\llbracket#1\rrbracket}

\newcommand{\mlab}{{\ell}}
\newcommand{\mcntr}{{\delta}}
\newcommand{\mtcntr}{{\epsilon}}


\begin{document}

\conferenceinfo{WXYZ '05}{date, City.} 
\copyrightyear{2005} 
\copyrightdata{[to be supplied]} 

% \titlebanner{banner above paper title}        % These are ignored unless
% \preprintfooter{short description of paper}   % 'preprint' option specified.

\title{Optimizing Abstract$^{\text 2}$ Machines}

%% \authorinfo{J. Ian Johnson}
%%            {Northeastern University}
%%            {ianj@ccs.neu.edu}
%% \authorinfo{Matthew Might}
%%            {University of Utah}
%%            {might@cs.utah.edu}
%% \authorinfo{David Van Horn}
%%            {Northeastern University}
%%            {dvanhorn@ccs.neu.edu}
\authorinfo{}{}{}

\maketitle

\begin{abstract}
Recently \emph{abstracting abstract machines} has been proposed as a
systematic approach to designing sound and computable program analyses
for higher-order programming languages.  The approach has been applied
to a wide variety of languages with features widely considered
difficult to analyze


The abstracting abstract machines approach to program analysis makes
it easy to design and implement sound and computable abstract
interpretations of higher-order programs.
%
However, the straightforward implementation is generally
not very fast.
%
This article contributes a step by step process for going from a naive
abstract$^2$ machine to an efficient program analyzer that is (nearly)
three orders of magnitude faster than the starting point, yet just as
precise (and purely functional).
\end{abstract}

%% \category{CR-number}{subcategory}{third-level}

%% \terms
%% term1, term2

%% \keywords
%% keyword1, keyword2

\section{Introduction}

\subsection{Notation and prerequisites}

Prereqs: Semantics Engineering \cite{dvanhorn:Felleisen2009Semantics},
AAM \cite{dvanhorn:VanHorn2011Abstracting}
\cite{dvanhorn:VanHorn2012Systematic}.

Notation: concrete examples of programs under analysis is given in
(monochrome) Scheme notation.  Code is given in (syntax colored)
Racket.

\newpage
\section{Starting point}
\subsection{An abstract$^2$ machine for ISWIM}

ISWIM is a family of programming languages parameterized by a set of
base values and operations.  To make things concrete, we consider a
member of the ISWIM family with integers, booleans, and a few
operations.

\begin{figure}
\[
\begin{array}{l@{\qquad}rcl}
\text{Expressions} & \mathit{e} &=& \svar[^\mlab]\mvar\\
&&|& \slit[^\mlab]\mlit\\
&&|& \slam[^\mlab]\mvar\mexp\\
&&|& \sapp[^\mlab]\mexp\mexp \\
&&|& \sif[^\mlab]\mexp\mexp\mexp \\
\text{Variables}&\mvar &=& \syntax{x}\ |\ \syntax{y}\ |\ \dots\\
\text{Literals}&\mlit &=& \mnum\ |\ \mbln\ |\ \mop\\
\text{Integers}&\mnum &=& \syntax{0}\ |\ \syntax{1}\ |\ \syntax{-1}\ |\ \dots\\
\text{Booleans}&\mbln &=& \strue\ |\ \sfalse\\
\text{Operations}&\mop &=& \syntax{zero?}\ |\ \syntax{add1}\ |\ \syntax{sub1}\ |\ \dots
\end{array}
\]
\caption{Syntax of ISWIM}
\label{fig:syntax}
\end{figure}

Figure~\ref{fig:syntax} defines the (abstract) syntax of ISWIM.
Figure~\ref{fig:aam} defines the semantics of ISWIM as a machine
model.  Evaluation is defined as the set of states reachable by the
reflexive, transitive closure of the machine transition relation.  The
machine is a very slight variation on a standard abstract machine for
ISWIM in ``eval, continue, apply'' form.

Compared with the standard machine semantics, this definition is
different in the following ways:
\begin{itemize}
\item the store maps addresses to \emph{sets} of values,
\item continuations are heap-allocated,
\item there are ``contour values'' (written $\mcntr$) and syntax
  labels ($\mlab$) threaded through the computation, and
\item the machine is implicity parameterized by the functions
  $\mathit{push}$, $\mathit{bind}$, and $\interpdelta$.
\end{itemize}

\begin{figure*}
\begin{align*}
\mathit{eval}(\mexp) &= \{ \mstate\ |\ \ev[^{\mtcntr}]{\mexp,\varnothing,\varnothing,\kmt} \multimachstep \mstate \} \text{ where }
\\[2mm]
%% EVAL
\ev{\svar\mvar,\menv,\msto,\mkont} &\machstep
\co{\mkont,\mval,\msto}
\text{ where }\mval \in \msto(\menv(\mvar))
\\
\ev{\slit\mlit,\menv,\msto,\mkont} &\machstep
\co{\mkont,\mlit,\msto}
\\
\ev{\slam\mvar\mexp,\menv,\msto,\mkont} &\machstep
\co{\mkont,\clos{\mvar,\mexp,\menv},\msto}
\\
\ev[^\mcntr]{\sapp[^\mlab]{\mexpi0}{\mexpi1},\menv,\msto,\mkont} &\machstep
\ev[^\mcntr]{\mexpi{0},\menv,\msto',\kar[_\mlab^\mcntr]{\mexpi{1},\menv,\maddr}}
\text{ where }\maddr,\msto' = \mathit{push}^\mcntr_\mlab(\msto,\mkont)
\\
\ev[^\mcntr]{\sif[^\mlab]{\mexpi0}{\mexpi1}{\mexpi2},\menv,\msto,\mkont} &\machstep
\ev[^\mcntr]{\mexpi0,\menv,\msto',\kif[^\mcntr]{\mexpi1,\mexpi2,\menv,\maddr}}
\text{ where }\maddr,\msto' = \mathit{push}_\mlab^\mcntr(\msto,\mkont)
\\[2mm]
%% CONTINUE
\co{\kmt,\mval,\msto} &\machstep
\ans{\msto,\mval}
\\
\co{\kar[^\mcntr_\mlab]{\mexp,\menv,\maddr},\mval,\msto} & \machstep
\ev[^\mcntr]{\mexp,\menv,\msto,\kfn[^\mcntr_\mlab]{\mval,\maddr}}
\\
\co{\kfn[^\mcntr_\mlab]{{\mvalx{u}},\maddr},\mval,\msto} & \machstep
\ap[^\mcntr_\mlab]{\mval,\mvalx{u},\mkont,\msto}
\text{ where }\mkont \in \msto(\maddr)
\\
\co{\kif[^\mcntr]{\mexpi0,\mexpi1,\menv,\maddr},\strue,\msto} & \machstep
\ev[^\mcntr]{\mexpi0,\menv,\msto,\mkont}
\text{ where }\mkont\in\msto(\maddr)
\\
\co{\kif[^\mcntr]{\mexpi0,\mexpi1,\menv,\maddr},\sfalse,\msto} & \machstep
\ev[^\mcntr]{\mexpi1,\menv,\msto,\mkont}
\text{ where }\mkont\in\msto(\maddr)
\\[2mm]
%% APPLY
\ap[^\mcntr_\mlab]{\clos{\mvar,\mexp,\menv},\mval,\msto,\mkont} & \machstep
\ev[^{\mcntr'}\!]{\mexp,\menv',\msto',\mkont}
\text{ where }\menv',\msto',\mcntr' = \mathit{bind\,}^{ \mcntr}_\mlab(\msto,\mvar,\mval)
\\
\ap[^\mcntr_\mlab]{\mop,\mval,\msto,\mkont} & \machstep
\co{\mkont,\mval',\msto}
\text{ where }\mkont\in\msto(\maddr)
\text{ and } \mval'\in\interpdelta(\mop,\mval)
\end{align*}
\caption{Abstract$^2$ machine for ISWIM}
\label{fig:aam}
\end{figure*}


\paragraph{Concrete interpretation} can be characterized by setting the implicit
parameters of the relation given in Figure~\ref{fig:aam} as follows:
\begin{align*}
\mathit{push}(\mlab,\msto,\mkont) &= \maddr,\msto\sqcup[\maddr\mapsto\{\mkont\}]
\mbox{ where }\maddr \notin\msto
\\
\mathit{bind}(\msto,\mvar,\mval) &= \menv[\mvar\mapsto\maddr],\msto\sqcup[\maddr\mapsto\{\mval\}]
\mbox{ where }\maddr \notin\msto
\end{align*}
The resulting relation is non-deterministic in its choice of
addresses, however it must always choose a fresh address when
allocating a continuation or variable binding.  If we consider machine
states equivalent up to consistent renaming, this relation defines
a deterministic machine.  (The relation is really a function.)


\paragraph{Abstract interpretation} can be characterized by setting the implicit
parameters just as above, but dropping the $\maddr \not\in \msto$
condition.  This family of interpereters is also non-deterministic in
choices of addresses, but it is free to choose addresses that are
already in use.  Consequently, the machines may be non-deterministic
when mutiple values reside in a store location.

It is important to recognize from this definition that \emph{any}
allocation strategy is an abstract interpretation.  In particular,
concrete intepretation is a kind of abstract interpretation.  So is an
interpretation that allocates a single cell into which all bindings
and continuations are stored.  On the one hand is an abstract
intepretation that is non-computable and gives only the ground truth
of a programs behavior; on the other is an abstract interpretation
that is easy to compute but gives little information.  Useful program
analyses lay somewhere in between and can be characterized by their
choice of address representation and allocation strategy.

%% We now have a framework for describing program analysis for the ISWIM
%% family of languages, whereby approximation of both control and
%% environment structure is regulated by the heap and allocation
%% policies.

Uniform \(k\)-CFA is one such analysis.

\paragraph{Uniform \(k\)-CFA} can be characterized by the following allocation
strategy:

\begin{align*}
\mathit{push} &= \dots\\
\mathit{bind} &= \dots
\end{align*}

\paragraph 0CFA

\begin{align*}
\mathit{push}(\mlab,\msto,\mkont) &= \mlab,\msto\sqcup[\mlab\mapsto\{\mkont\}]
\\
\mathit{bind}(\msto,\mvar,\mval) &= \menv[\mvar\mapsto\mvar],\msto\sqcup[\mvar\mapsto\{\mval\}]
\end{align*}


\paragraph{Primitives}



%% \begin{align*}
%% \widehat{\mathit{push}}(\maddr,\msto,\mkont) &= \maddr,\msto\sqcup[\maddr\mapsto\{\mkont\}]
%% % \mbox{ where }\maddr \notin\msto
%% \\
%% \widehat{\mathit{bind}}(\msto,\mvar,\mval) &= \menv[\mvar\mapsto\maddr],\msto\sqcup[\maddr\mapsto\{\mval\}]
%% \end{align*}


\begin{align*}
\mnum+1 &\in \interpdelta(\saddone,\mnum) &
\mnum-1 &\in \interpdelta(\ssubone,\mnum)\\
\strue &\in \interpdelta(\szerohuh,\szero) &
\sfalse &\in \interpdelta(\szerohuh,\mnum)\text{ if }\mnum\neq \szero\\
\end{align*}

\begin{align*}
\sNum &\in \hat\interpdelta(\saddone,\mnum) &
\sNum &\in \hat\interpdelta(\ssubone,\mnum)\\
\strue &\in \hat\interpdelta(\szerohuh,\sNum) &
\sfalse &\in \hat\interpdelta(\szerohuh,\sNum)\\
\strue &\in \hat\interpdelta(\szerohuh,\szero) &
\sfalse &\in \hat\interpdelta(\szerohuh,\mnum)\text{ if }\mnum\neq \szero\\
\end{align*}

HERE: ρ

\begin{figure}
\begin{alltt}
  ;; State \(\rightarrow\) Setof State
  (define (step \(\mstate\))
    (match \(\mstate\)
      [(ev \(\mexp\) \(\rho\) \(\msto\) \(\mkont\))
       (match \(\mexp\)
         [(var\(\superscript\mlab\) \(\mvar\))
          (for/set ((\(\mval\) (lookup \(\rho\) \(\msto\) \(\mvar\))))
            (co \(\msto\) \(\mkont\) \(\mval\)))]
         [(lit\(\superscript\mlab\) \(\mlit\)) (set (co \(\msto\) \(\mkont\) \(\mlit\)))]
         [(lam\(\superscript\mlab\) \(\mvar\) \(\mexp\)) (set (co \(\msto\) \(\mkont\) (clos \(\mvar\) \(\mexp\) \(\rho\))))]
         [(app\(\superscript\mlab\) \(\mexp\subscript0\) \(\mexp\subscript1\))
          (define-values (\(\msto'\) \(\maddr\)) (push \(\mstate\)))
          (set (ev \(\mexp\subscript0\) \(\rho\) \(\msto'\) (ar \(\mexp\subscript1\) \(\rho\) \(\maddr\))))]
         [(ife\(\superscript\mlab\) \(\mexp\subscript0\) \(\mexp\subscript1\) \(\mexp\subscript2\))
          (define-values (\(\msto'\) \(\maddr\)) (push \(\mstate\)))
          (set (ev \(\mexp\subscript0\) \(\rho\) \(\msto'\) (ifk \(\mexp\subscript1\) \(\mexp\subscript2\) \(\rho\) \(\maddr\))))])]
      [(co \(\msto\) \(\mkont\) \(\mval\))
       (match \(\mkont\)
         ['mt (set (ans \(\msto\) \(\mval\)))]
         [(ar\(\superscript\mlab\) \(\mexp\) \(\rho\)) (set (ev \(\mexp\) \(\rho\) \(\msto\) (fn \(\mval\) l)))]
         [(fn\(\superscript\mlab\) \(\mval'\))
          (for/set ((\(\mkont\) (get-cont \(\msto\) l)))
            (ap \(\msto\) \(\mval'\) \(\mval\) \(\mkont\)))]
         [(fi\(\superscript\mlab\) c a \(\rho\))
          (for/set ((k (get-cont \(\msto\) l)))
            (ev (if v c \(\maddr\)) \(\rho\) \(\msto\) \(\mkont\)))])]
      [(ap \(\msto\) fun \(\maddr\) \(\mkont\))
       (match fun
         [(clos l \(\mvar\) \(\mexp\) \(\rho\))
          (define-values (\(\rho'\) \(\msto'\)) (bind \(\mstate\)))
          (set (ev \(\msto'\) \(\mexp\) \(\rho'\) \(\mkont\)))]
         [(op \(\mop\))
          (for*/set ((\(\mkont\) (get-cont \(\msto\) l))
                     (\(\mval\) (\(\interpdelta\) \(\mop\) \(\mval\))))
            (co \(\msto\) \(\mkont\) \(\mval\)))]
         [_ (set)]))]))
\end{alltt}
\caption{Implementation of machine transition relation.}
\end{figure}

\subsection{An example: Analyzing Church number computations}

\newcommand{\church}[1]{\(\ulcorner{\tt #1}\urcorner\)}

%% \begin{alltt}
%% (define \church2 (\(\lambda\) (f) (\(\lambda\) (x) (f (f x)))))

%% (define pred
%%   (\(\lambda\) (n)
%%     (\(\lambda\) (rf)
%%       (\(\lambda\) (rx)
%%         (((n (\(\lambda\) (g) (\(\lambda\) (h) (h (g rf)))))
%%           (\(\lambda\) (i) rx))
%%          (\(\lambda\) (id) id))))))
%% \end{alltt}


%% (define \church0 (\(\lambda\) (f0) (\(\lambda\) (x0) x0)))
%% (define \church1 (\(\lambda\) (f1) (\(\lambda\) (x1) (f1 x1))))
%% (define \church2 (\(\lambda\) (f2) (\(\lambda\) (x2) (f2 (f2 x2)))))
%% (define \church3 (\(\lambda\) (f3) (\(\lambda\) (x3) (f3 (f3 (f3 x3))))))

\begin{figure}
\begin{alltt}
(define ((((plus p1) p2) pf) x)
  ((p1 pf) ((p2 pf) x)))

(define (((mult m1) m2) mf)
  (m2 (m1 mf)))

(define (((pred n) rf) rx)
  (((n (\(\lambda\) (g) (\(\lambda\) (h) (h (g rf)))))
    (\(\lambda\) (ignored) rx))
   (\(\lambda\) (id) id)))

(define (zero? n)
  ((n (\(\lambda\) (zx) #f)) #t))

(define ((church=? e1) e2)
  (if (zero? e1) (zero? e2)
      (if (zero? e2) #f
          ((church=? (pred e1)) (pred e2)))))

;; multiplication distributes over addition
((church=? ((mult \church2) ((plus \church1) \church3)))
 ((plus ((mult \church2) \church1)) ((mult \church2) \church3)))
\end{alltt}
\caption{Church-numeral calculation}
\end{figure}

Where \syntax{church=?} is an equality function for Church numerals
written in terms of recursion and an iterative \syntax{pred} function.
The \syntax{plus} and \syntax{mult} functions are written as usual.

This program reaches $N$ states under concrete interpretation.

\newpage
\subsection{Generic fixpoint calculator}

\begin{alltt}
;; Expr \(\rightarrow\) Setof State
(define (eval e)
  (fix step (set (inj e))))
\end{alltt}

We compute the semantics of a program by iterating the state
transition relation until a fixed point in the reachable states is
reached.

\begin{verbatim}
;; appl : (∀ (X) ((X -> (Setof X)) -> ((Setof X) -> (Setof X))))
(define ((appl f) s)
  (for/fold ([i (set)])
    ([x (in-set s)])
    (set-union i (f x))))

;; Calculate fixpoint of (appl f).
;; fix : (∀ (X) ((X -> (Setof X)) (Setof X) -> (Setof X)))
(define (fix f s)
  (let loop ((accum (set)) (front s))
    (if (set-empty? front)
        accum
        (let ((new-front ((appl f) front)))
          (loop (set-union accum front)
                (set-subtract new-front accum))))))
\end{verbatim}

\subsection{Store widening}

\begin{verbatim}
;; State^ = (cons (Set Conf) Store)

;; (State -> Setof State) -> State^ -> { State^ }
(define ((wide-step step) state)
  (match state
    [(cons cs \(\msto\))
     (define ss ((appl step)
                 (for/set ([c cs]) (c->s c \(\msto\)))))
     (set (cons (for/set ([s ss]) (s->c s))
                (join-stores ss)))]))
\end{verbatim}

\subsection{Baseline evaluation}

Wide Store: cpu time: 551571 real time: 571319 gc time: 4003

\section{Precision preserving recipe}

The first four techniques reside in the purely functional realm. The
following three techniques use imperative constructs to avoid allocation
overhead.

\subsection{Lazy non-determinism}

Lazy:
   cpu time: 32481 real time: 32881 gc time: 547

\subsection{Abstract compilation}

We can eliminate the {\tt ev} states from the run-time interpretation
of a program by specializing the machine transition relation to the
program being analyzed.  This eliminates interpretative overhead by
first compiling the program into ``byte code'' instructions.

The essence of the compilation effect can be seen by consider an example
such as
\[
\sapp{\sapp{\sapp\mvar{\mexp_1}}{\mexp_2}}{\mexp_3}
\]
which makes the following transitions:
\begin{align}
& \ev{\sapp{\sapp{\sapp\mvar{\mexp_1}}{\mexp_2}}{\mexp_3},\menv,\mkont,\msto_0}\\
\machstep\; &
\ev{\sapp{\sapp\mvar{\mexp_1}}{\mexp_2},\menv,\kar{\mexp_3,\menv,\maddr_1},\msto_1}
\\
\machstep\; &
\ev{\sapp\mvar{\mexp_1},\menv,\kar{\mexp_2,\menv,\maddr_2},\msto_2}
\\
\machstep\; &
\ev{\mvar, \menv,\kar{\mexp_1,\menv,\maddr_3},\msto_3} % {\mexp_2}
\\
\machstep\; &
\co{\kar{\mexp_1,\menv},\mval,\msto_4} % {\mexp_1}{\mexp_2}
\mbox{ where } \mval \in \msto(\menv(\maddr))
\end{align}

where $\msto_4 = \msto_0 \sqcup \{ [\maddr_1 \mapsto \{ \mkont \}],
[\maddr_2 \mapsto \kar{\mexp_3,\menv,\maddr_1}]
[\maddr_3 \mapsto \kar{\mexp_2,\menv,\maddr_2}]$.


Notice that the only difference between concrete and abstract interpretation
is which addresses are pushed.

\begin{figure}
\begin{align*}
\compile{\svar\mvar} &= \lambda(\menv,\msto,\mkont) .\co{\mkont,\mval,\msto} \text{ where }\mval\in\msto(\menv(\mvar))
\\
\compile{\slit\mlit} &= \lambda(\menv,\msto,\mkont) .
\co{\mkont,\mlit,\msto}
\\
\compile{\slam\mvar\mexp} &= \lambda(\menv,\msto,\mkont) .
\co{\mkont,\clos{\mvar,\compile\mexp,\menv},\msto}
\\
\compile{\sapp[^\mlab]{\mexpi0}{\mexpi1}} &= \lambda^\mcntr(\menv,\msto,\mkont) .
\compile{\mexpi0}^\mcntr(\menv,\msto',\kar[_\mlab^\mcntr]{\compile{\mexpi1},\menv,\maddr})
\\
&
\text{ where }\maddr,\msto' = \mathit{push}^\mcntr_\mlab(\msto,\mkont)
\\
\compile{\sif[^\mlab]{\mexpi0}{\mexpi1}{\mexpi2}} &= \lambda^\mcntr(\menv,\msto,\mkont) .
\compile{\mexpi0}^\delta(\menv,\msto',\kif[^\mcntr]{\compile{\mexpi1},\compile{\mexpi2},\menv,\maddr})
\\
&\text{ where }\maddr,\msto' = \mathit{push}_\mlab^\mcntr(\msto,\mkont)
\end{align*}
\caption{Compilation}
\end{figure}

\begin{figure}
\begin{align*}
\mathit{eval}(\mexp) &= \{ \mstate\ |\ \compile{\mexp}(\epsilon,\varnothing,\varnothing,\kmt) \multimachstep \mstate \} \text{ where }
\\[2mm]
%% CONTINUE
\co{\kmt,\mval,\msto} &\machstep
\ans{\msto,\mval}
\\
\co{\kar[^\mcntr_\mlab]{\mcomp,\menv,\maddr},\mval,\msto} & \machstep
\mcomp^\mcntr(\menv,\msto,\kfn[^\mcntr_\mlab]{\mval,\maddr})
\\
\co{\kfn[^\mcntr_\mlab]{{\mvalx{u}},\maddr},\mval,\msto} & \machstep
\ap[^\mcntr_\mlab]{\mval,\mvalx{u},\mkont,\msto}
\text{ where }\mkont \in \msto(\maddr)
\\
\co{\kif[^\mcntr]{\mcompi0,\mcompi1,\menv,\maddr},\strue,\msto} & \machstep
\mcompi0^\mcntr(\menv,\msto,\mkont)
\text{ where }\mkont\in\msto(\maddr)
\\
\co{\kif[^\mcntr]{\mcompi0,\mcompi1,\menv,\maddr},\sfalse,\msto} & \machstep
\mcompi1^\mcntr(\menv,\msto,\mkont)
\text{ where }\mkont\in\msto(\maddr)
\\[2mm]
%% APPLY
\ap[^\mcntr_\mlab]{\clos{\mvar,\mcomp,\menv},\mval,\msto,\mkont} & \machstep
\mcomp^{\mcntr'}(\menv',\msto',\mkont)
\text{ where }\menv',\msto',\mcntr' = \mathit{bind\,}^{ \mcntr}_\mlab(\msto,\mvar,\mval)
\\
\ap{\mop,\mval,\msto,\mkont} & \machstep
\co{\mkont,\mval',\msto}
\text{ where }\mkont\in\msto(\maddr)
\text{ and } \mval'\in\interpdelta(\mop,\mval)
\end{align*}
\caption{Abstract$^2$ machine for compiled ISWIM}
\label{fig:caam}
\end{figure}


Compile:
   cpu time: 255397 real time: 261532 gc time: 2947

\noindent
Compile + Lazy:
   cpu time: 31173 real time: 31642 gc time: 739

\newpage
\subsection{Fixed-point specialization}

\begin{alltt}
;; State^ -> State^
;; Specialized from wide-step : State^ -> { State^ } ≈ State^ -> State^
(define (wide-step-specialized state)
  (match state
    [(cons \(\msto\) cs)
     (define-values (cs* \(\msto\)*)
       (for/fold ([cs* (set)] [\(\msto\)* \(\msto\)])
         ([c cs])
         (match (step-compiled^ (cons \(\msto\) c))
           [(cons \(\msto\)** cs**)
            (values (set-union cs* cs**) (join-store \(\msto\)* \(\msto\)**))])))
     (cons \(\msto\)* (set-union cs cs*))]))
\end{alltt}
%% Should be commutted to beginning

Special + Compile + Lazy:
   cpu time: 14212 real time: 14681 gc time: 823

\subsection{Computing with store changes}

\begin{alltt}
;; State^ -> State^
;; Specialized from wide-step : State^ -> { State^ } ≈ State^ -> State^
(define (wide-step-specialized state)
  (match state
    [(cons \(\msto\) cs)
     (define-values (cs* ∆)
       (for/fold ([cs* (set)] [∆* '()])
         ([c cs])
         (match (step-compiled^ (cons \(\msto\) c))
           [(cons \(\msdiff\)** cs**)
            (values (set-union cs* cs**) (append \(\msdiff\)** \(\msdiff\)*))])))
     (cons (update \(\msdiff\) \(\msto\)) (set-union cs cs*))]))
\end{alltt}

\[
\{ \widehat{v_3}, \widehat{v_5}, \widehat{v_6} \} = 0001011
\]

\[
\sigma = \mbox{\tt \#(0$_0$,\dots,0$_{|P|}$)}
\]

Delta Store + Special + Compile + Lazy:
   cpu time: 668 real time: 686 gc time: 41

% For David to fill in.
\subsection{Imperative global store and worklist}
...
\subsection{Pre-allocated global store}
...

% Ian's sections: generators, implementation and compound data
\section{Implementation}

All the above techniques can be combined into a single implementation
of the small step reduction relation and small variations of a general
fixpoint combinator. We implementated all of these ideas in an
analysis of R4RS scheme in just 3000 lines of Racket. Most lines are
spent listing all the primitives and library functions (and their
abstract and concrete implementations) of R4RS. Here we describe the
unifying vocabulary for writing such a framework, and some design
decisions related to scaling to a language with compound data.

\subsection{Vocabulary of Parameterized Optimizations}

We combine the interface of generators~\cite{ianjohnson:cluhistory}
with a special lambda form ({\tt $\lambda$\%}) that wraps the meanings
of {\tt ev} states. We also have a special {\tt do} form that performs
behind-the-scenes state passing and iteration much like {\tt do} in
Haskell. In the presence of the different optimization
techniques, {\tt generator}, {\tt yield}, {\tt $\lambda$\%} and {\tt do}
change meaning slightly, but keep the same interface as their names
suggest.

\begin{figure}
\begin{alltt}
\(e-dispatch =\)
(match e
  [(var\(\superscript\mlab\) x)
   (\(\lambda\)\% (\(\msto\) \(\menv\) k)
    (do (\(\msto\)) ((v (delay (lookup-env \(\menv\) x))))
      (yield (co \(\msto\) k v))))]
  [(lit\(\superscript\mlab\) l)
   (\(\lambda\)\% (\(\msto\) \(\menv\) k)
    (do (\(\msto\)) () (yield (co \(\msto\) k n))))]
  [(lam\(\superscript\mlab\) x e)
   (define c (compile e))
   (\(\lambda\)\% (\(\msto\) \(\menv\) k)
    (do (\(\msto\)) () (yield (co \(\msto\) k (clos x c \(\menv\))))))]
  [(app\(\superscript\mlab\) f e)
   (define cf (compile f))
   (define ce (compile e))
   (\(\lambda\)\% (\(\msto\) \(\menv\) k)
    (define-values (\(\msto\)* a) (push state))
    (yield (ev \(\msto\)* cf \(\menv\) (ar c \(\menv\) a))))]
  [(ife\(\superscript\mlab\) e0 e1 e2)
   (define c0 (compile e0))
   (define c1 (compile e1))
   (define c2 (compile e2))
   (\(\lambda\)\% (\(\msto\) \(\menv\) k)
     (do (\(\msto\)) [(\(\msto\)* a) (push state)]
      (yield (ev \(\msto\)* c0 \(\menv\) (ifk c1 c2 \(\menv\) a)))))])

(define (step state)
  (match state
    [(ev \(\msto\) e \(\menv\) k) \(e-dispatch\)] ;; dead if compiled.
    [(co \(\msto\) k v)
     (match k
       ['mt (generator (do (\(\msto\)) ([v <- (force \(\msto\) v)])
        (yield (ans \(\msto\) v))))]
       [(ar\(\superscript\mlab\) e \(\menv\))
        (generator (do (\(\msto\)) ()
          (yield (ev \(\msto\) e \(\menv\) (fn v l)))))
       [(fn\(\superscript\mlab\) f)
        (generator (do (\(\msto\)) ([k <- (get-cont \(\msto\) l)])
         (yield (ap \(\msto\) f v k))))]
       [(fi\(\superscript\mlab\) c a \(\menv\))
        (generator
         (do (\(\msto\)) ([k <- (get-cont \(\msto\) l)]
                  [v <- (force \(\msto\) v)])
           (yield (ev \(\msto\) (if v c a) \(\menv\) k))))])]
    [(ap \(\msto\) fun a k)
     (match fun
       [(clos l x e \(\menv\))
        (generator
         (do (\(\msto\)) ([(\(\menv\)* \(\msto\)*) (bind state)])
           (yield (ev \(\msto\)* e \(\menv\)* k))))]
       [(? op? o)
        (do (\(\msto\)) ([k <- (get-cont \(\msto\) l)]
                 [v <- (force \(\msto\) v)]
                 [v (\(\interpdelta\) o (list v))])
          (yield (co \(\msto\) k v)))
       [_ (generator (do (\(\msto\)) () (continue)))]))]))
\end{alltt}
\caption{Parameterized Abstract$^2$ machine for ISWIM}
\label{fig:paam}
\end{figure}

\begin{figure}
\begin{alltt}
(define (compile e) \(e-dispatch\))
(\(\lambda\)\% (\(\msto\) \(\menv\) k) body \ldots) \(\machstep\) (\(\lambda\) (\(\msto\) \(\menv\) k) body \ldots)
(yield (ev \(\msto\) c . rest)) \(\longmapsto\) (c \(\msto\) . rest)
(yield e) \(\longmapsto\) (yield-meaning e)
\end{alltt}
\caption{Meaning of forms if compiled}
\label{fig:cfm}
\end{figure}

\begin{figure}
\begin{alltt}
(define (compile e) e)
(\(\lambda\)\% (\(\msto\) \(\menv\) k) body \ldots) \(\longmapsto\) (let () body ...)
(yield e) \(\longmapsto\) (yield-meaning e)
\end{alltt}
\caption{Meaning of forms if not compiled}
\label{fig:ncfm}
\end{figure}

Why generators? Relations in general cannot be written as computable
functions, but in our case we have a decidable relation with finitely
many pairs sharing the same left component. There are several
implementation strategies for such relations, some more effective than
others. For instance, a silly implementation strategy would be
defining a binary function that decides membership in the relation. We
want the definition to drive the interpreter, so we might naturally
decide to write a function that takes a state and returns a set of all
the next states. As we described in the imperative workset section,
this intermediate allocation is unnecessary and wasteful. What we
really want is a way of pulling on the reduction relation for more
next states as the fixpoint is ready for them.

For intermediate sets, {\tt yield} is {\tt return} in the set
monad. For the imperative worklist, {\tt yield} performs mutation
behind the scenes. Indeed, we could use real generators! This is the
only case where {\tt generator} constructs a nullary generator; in the
other cases it evaluates to its body. Racket's generators are not
tuned for performance, so we actually lose efficiency in this case,
but the unifying vocabulary is instructive nonetheless.

The {\tt $\lambda$\%} form constructs a real closure in the
compilation case and evaluates to its body otherwise.

The {\tt do} form has the special job of maintaining the current store
and accumulated states (if indeed those are being passed around). It
uses Racket's syntax parameters ~\cite{ianjohnson:eli/stxparam} to
track the most recent bindings of these accumulators so that {\tt
  yield-meaning} can refer to them (say {\tt $\sigma$p} and {\tt
  statep}). In the case of state passing and set-accumulated states, its
definition is simply

\begin{center}
\begin{alltt}
(yield-meaning e) \(\longmapsto\) (values \(\sigma\)p (set-add statep e))
\end{alltt}
\end{center}

The linguistic elements described above were all reusable to implement
the various languages' abstract interpreters. One such language was
R4RS Scheme, which has compound data (e.g. vectors and conses). 

\subsection{Abstracting Compound Data}

We discovered that the choice of abstract domain for literal data
(such as the large quoted axiom database in the Boyer benchmark)
drastically affected the performance of our analysis on the benchmark
suite. It is worth discussing the ``obvious'' and na\"ive abstraction
that is natural to a reader of
~\citep{dvanhorn:VanHorn2011Abstracting}, and other easy-to-implement
yet effective approaches.

\subsubsection{Na\"ively}

The uniform way they approach a simple abstraction strategy is to cut
recursion out of the data definition by tying the recursive knot
through the abstract store. For Scheme, the grammar for values looks like the following:

\begin{alltt}
Value ::= #t | #f | (cons Value Value) | '() | ...
\end{alltt}

Upon evaluating a {\tt cons} application, we instead allocate two
addresses $a$ and $d$, join them to the respective values in the
store, and return the flattened {\tt (cons $a$ $d$)} value. Since
these addresses are all distinguished at different syntactic
callsites, and quoted lists really are sugar for a sequence of calls
to {\tt cons}, this abstraction explodes the value space. Analyzing a
function that counts the number of atoms in a literal s-expression
would actually interpret that function at least that number of times
(more because of intermediate conses). Indeed, even in our fastest
parameterization, this abstraction causes the analysis of Boyer to be
430 times slower than the approach we will now describe.

\subsubsection{Less na\"ively and less precisely}

The number of syntactic uses of {\tt cons} versus implicit uses via
literal lists is dramatically smaller in typical Scheme programs. We
use the above abstraction for these syntactic uses, but choose a
different strategy for quoted lists. Quotation is special because it
cannot introduce function values or mutable values, which is important
to enhancing our technique's soundness to conceptual complexity ratio.

There are a few steps to consider:
\begin{itemize}
 \item{Define special value lattice elements for compound data domains that can be quoted
       (e.g. {\tt QPair} for $\left\lbrace ({\tt cons} a_i d_i)\right\rbrace_i$, {\tt QVector} for immutable vectors, etc.)}
 \item{Define a ``larger'' value lattice element for all quotable data, {\tt QData}}
 \item{Interpret {\tt (quote (a ...))} as {\tt (qlist a ...)}, a new primitive function defined in figure \ref{fig:qlist},
       and similar definitions for immutable vectors.}
 \item{Extend the $\interpdelta$ axioms to include conservative meaning for these new values (e.g. {\tt (car QData)} = {\tt QData}, {\tt (add1 QData)} = {\tt Number} and log ``possible type error'') and allow them to allocate addresses and change the store}
\end{itemize}

\begin{figure}
\begin{align*}
\interpdelta(\msto, qlist) &= (\mbox{{\tt '()}}, \msto) \\
\interpdelta(\msto, qlist, v ..._+) &= (({\tt cons}\ a\ d), \msto') \\
\qquad \mbox{where } \msto' &= \msto\sqcup[a \mapsto \bigsqcup(v ...)] \\
                            &\qquad \sqcup[d \mapsto \lbrace \mbox{{\tt '()}}, ({\tt cons}\ a\ d)\rbrace] \\
\bigsqcup(v) &= v \\
\bigsqcup(v, vs ...) &= merge(v, \bigsqcup(vs ...)) \\
merge(v, v) &= v \\
merge(n, m) &= {\tt Number} \\
merge(n, v) &= {\tt QData} \\
merge(({\tt cons}\ a\ d), ({\tt cons}\ a'\ d')) &= {\tt QPair} \\
merge({\tt QPair}, ({\tt cons}\ a\ d)) &= {\tt QPair} \\
merge({\tt QPair}, v) &= {\tt QData} \\
\vdots
\end{align*}
\caption{Quoted list primitive}
\label{fig:qlist}
\end{figure}

The alternative to {\tt QData} is to go higher in the lattice to
$\top$, which has more complicated meaning (Shivers' escape semantics) and
is overly approximate.

\section{Evaluation}

\cite{dvanhorn:Earl2012Introspective}

\cite{dvanhorn:wright-jagannathan-toplas98}

Other benchmarks

\section{Related work}

Boucher and Feeley \cite{dvanhorn:Boucher1996Abstract} introduced the
idea of \emph{abstract compilation}, which used closure generation
\cite{dvanhorn:Feeley1987Using} to improve the performance of control
flow analysis.  We have adapted the closure generation technique from
composition evaluators to abstract machines and applied it to similar
effect.

\section{Conclusion}

Abstract machines are not only a good model for rapid analysis
development, they also can be systematically developed into efficient
algorithms.

\bibliographystyle{splncs}
\bibliography{bibliography}

\appendix
\section{Relation to Uniform \(k\)-CFA (A Case Against Acceptability)}

\cite{dvanhorn:nielson-nielson-popl97} \cite{dvanhorn:Neilson:1999}

This machine's allocation strategy mimics the Uniform k-CFA analysis
in Principles of Program Analysis, which is defined in terms of
``$\delta$ contours.''  However, because the machine represetation makes
context explicit via continuations, we can calculate these contours
rather than thread them throught the evaluator.  In other words, we
can use the CESK* machine without modification to obtain Uniform k-CFA
by way of a simple allocation strategy.  (In this way, it's a
simplification of the presentation in JFP.)

NNH uses a coinductive acceptability relation to specify Uniform
k-CFA:

\[
   C,R \models^{ce}_\delta E
\]

The cache and global environment form a finite store-like structure
holding bindings and return values.  The contour environment ce maps
variables to locations in R which contains their bindings, just as the
environment of the CESK* machine does.  The current contour delta is a
string of application labels describing the enclosing context under
which this term is being analyzed (or evaluated).  If you view the
acceptability relation as a big-step evalator, the
$(\widehat C,\widehat\rho)$ component should be seen as a global
store ce is the environment mapping variables to their locations.

Starting form the initial configuration for a program and iterating
the machine transition relation until reaching a fixpoint of reachable
states will \emph{underestimate} the acceptability relation of Uniform
k-CFA.  You can recover acceptability by feeding this store back into
the initial configuration and iterating again.  Repeating this process
until a complete run of the program reaches no new states will be the
least solution that is acceptable.

HOWEVER.  Why should we care about acceptability?  What this
machine computes is safe.  In other words, it computes a more
precise characterization of the run-time behavior of a program.  In
doing is so, it actually saves work (as can be seen above).

An Example:

\begin{verbatim}
 (let ((id (\(\lambda\) (x) x)))
   (begin (id 1) (id 2)))
\end{verbatim}

Under Uniform 0-CFA, we would have:
\[
   [{\tt x} \mapsto \{{\tt 1}, {\tt 2}\}] \in \widehat\rho
\]

in the least solution to $\models$.  This says that, when run, 'x' is
bound to 1 or 2.

Under the machine semantics using a 0CFA allocation policy, the trace
semantics of the machine show that x is bound to 1, and that at some
later point, x becomes bound to 1 or 2.  Moreover, the machine would
show that (id 1) evaluates to 1 and only 1, while Uniform 0CFA must
give that (id 1) is either 1 or 2 to be acceptable.  We don't see any
value in these kinds of false flows that are due to the global and
timeless aspects of C,R which acceptability requires the heap to be
both finite and unchanging over the course of abstract
interpretation. (Another view of the difference: the machine abstracts
a program's execution as a \emph{finite state machine} that mimics the
machine interpretation of the program; the aceptability relation of
Uniform \(k\)-CFA abstracts a program's execution as a \emph{finite
  map} that mimics the big-step evaluator: from terms to (sets of)
values.)


\subsection{Another problem with acceptability: Temporal ignorance}

The small-step approach to static analysis brings subtle yet important temporal
richness not found in classical analyses for higher-order programs.
%
Classical analyses (ultimately) compute judgments on program terms and
contexts, e.g., at expression $e$, value $x$ may have value $v$.
%
The judgments do not relate the order in which expressions and context may be
evaluated in a program, e.g., a classical analysis has nothing to say with
regard to question like, ``Do we always evaluate $e_1$ before $e_2$?'' or ``Is
it always the case that a file handle is opened, read and then closed in that
order?''

Small-step analyses, by their nature, encode the temporal relationships between
abstract states.
%
It is sensible to make temporal queries of a small-step analysis.
%
Of course, this does not come for free: respecting temporal order imposes an
order in which states and terms may be evaluated \emph{during} the analysis.
%
Classical analyses can (and do) evaluate expressions in any order, or in some
cases, even in parallel~\cite{might:Prabhu:2010:EigenCFA}.
%
Relaxing that restriction on order affords additional optimizations that we
have \emph{not} performed.

We avoid sacrificing order not simply because we are interested in the
questions it allows us to ask, but because considering temporal order actually
improves the precision of the analysis itself.



\section{Pushdown Analysis}

It is straightforward to instantiate a \emph{pushdown} abstraction by
bounding only the variable binding portion of the heap, but using a
unique allocation strategy for continuations.  Such a strategy
abstracts a program's execution as a \emph{pushdown automata}
that mimics the machine interpretation.  This strategy therefore
models the abstract stack in a true stack like fashion and always
properly matches function calls with their return.

Although such analyses can be formulated straightforwardly in the
abstract machine approach, it is not clear all of the techniques of
this paper can be applied to similar effect in the pushdown context.
The main problem is calculation of an analysis can no longer be
computed as the fixed point of the machine transition relation.
Although there are several implementations (CFA2,ICFP'12), they
operate at speeds roughly on par with our starting point: unoptimized
store widened
machines. \cite{dvanhorn:Earl2012Introspective,dvanhorn:Vardoulakis2011CFA2}

\end{document}
