\documentclass{llncs}
\usepackage{amsmath,amssymb,pfsteps,multicol,stmaryrd,mathpartir,centernot}
\newcommand{\superscript}[1]{\ensuremath{^{#1}}}
\newcommand{\subscript}[1]{\ensuremath{_{#1}}}
\newcommand{\tuple}[3][\ ]{{\tt #2{#1}}({#3})}

% Values
\newcommand{\clos}[1]{\tuple{clos}{#1}}
\newcommand{\rlos}[1]{\tuple{rlos}{#1}}

% States
\newcommand{\ev}[2][\ ]{\tuple[#1]{ev}{#2}}
\newcommand{\co}[1]{\tuple{co}{#1}}
\newcommand{\ap}[2][\ ]{\tuple[#1]{ap}{#2}}
\newcommand{\ans}[1]{\tuple{ans}{#1}}

% Continuations
\newcommand{\kmt}{\tt mt}
\newcommand{\kar}[2][\ ]{\tuple[#1]{ar}{#2}}
\newcommand{\kfn}[2][\ ]{\tuple[#1]{fn}{#2}}
\newcommand{\kif}[2][\ ]{\tuple[#1]{fi}{#2}}
\newcommand{\kuop}[2][\ ]{\tuple[#1]{oa}{#2}}
\newcommand{\kbopa}[2][\ ]{\tuple[#1]{oa1}{#2}}
\newcommand{\kbopb}[2][\ ]{\tuple[#1]{oa2}{#2}}

% Implementation forms
\newcommand{\generator}{{\tt generator}}
\newcommand{\yield}[1]{{\tt yield} #1}

% Syntax
\newcommand{\syntax}[1]{{\tt #1}}
\newcommand{\sapp}[3][\ ]{\tuple[#1]{app}{#2,#3}}
\newcommand{\slam}[3][\ ]{\tuple[#1]{lam}{#2,#3}}
\newcommand{\srec}[4][\ ]{\tuple[#1]{rec}{#2,#3,#4}}
\newcommand{\svar}[2][\ ]{\tuple[#1]{var}{#2}}
\newcommand{\snum}[2][\ ]{\tuple[#1]{num}{#2}}
\newcommand{\sbln}[2][\ ]{\tuple[#1]{bool}{#2}}
\newcommand{\sif}[4][\ ]{\tuple[#1]{if}{#2,#3,#4}}
\newcommand{\sop}[2][\ ]{\tuple[#1]{op}{#2}}
\newcommand{\sopu}[3][\ ]{\tuple[#1]{op}{#2,#3}}
\newcommand{\sopb}[4][\ ]{\tuple[#1]{op2}{#2,#3,#4}}
\newcommand{\strue}{{\tt tt}}
\newcommand{\sfalse}{{\tt ff}}
\newcommand{\saddone}{{\tt add1}}
\newcommand{\ssubone}{{\tt sub1}}
\newcommand{\szerohuh}{\syntax{zero?}}
\newcommand{\szero}{\syntax{0}}
\newcommand{\slit}[2][\ ]{\tuple[#1]{lit}{#2}}

\newcommand{\sNum}{\syntax{Z}}

% Metavariables
\newcommand{\maddr}{a}
\newcommand{\mvar}{x}
\newcommand{\mvarf}{f}
\newcommand{\mexp}{e}
\newcommand{\mexpi}[1]{e_{#1}}
\newcommand{\mexpf}{f}
\newcommand{\menv}{\rho}
\newcommand{\mkont}{\kappa}
\newcommand{\msto}{\sigma}
\newcommand{\mop}{o}
\newcommand{\mval}{v}
\newcommand{\mnum}{z}
\newcommand{\mbln}{b}
\newcommand{\mvalx}[1]{#1}
\newcommand\machstep{\longmapsto}
\newcommand\multimachstep{\longmapsto\!\!\!\!\!\rightarrow}
\newcommand{\mlit}{l}
\newcommand{\mstate}{\varsigma}
\newcommand{\mcomp}{k}
\newcommand{\mcompi}[1]{\mcomp_{#1}}
\newcommand{\interpdelta}{\Delta}

\newcommand{\compile}[1]{\llbracket#1\rrbracket}

\newcommand{\mlab}{{\ell}}
\newcommand{\mcntr}{{\delta}}
\newcommand{\mtcntr}{{\epsilon}}


\begin{document}

\section{Wide semantics decisions and precision differences}

There are a few straightforward ways to widen the state space to use a
single, monotonically increasing store. By far the most common way is
with a worklist algorithm for computing a least fixed point of a
``frontier-based'' collecting semantics. By dropping all the
intermediate stores, we lose temporal information such as $x$ is never
a number before $y$ is a string. This kind of information is
interesting, e.g., for diagnosing possible errors that the analysis
detects when $x$ is used in a call not defined for numbers. More
usefully, this can help pinpoint what influenced execution that lead
to a tainted value in an information flow analysis. Consider the
following program:

\begin{lstlisting}
(define (f x y)
  (if (bytes? y)
      (f (string->number x) (string-append (bytes->string/utf-8 y) x))
      (if (zero? x)
          y
          (f (sub1 x) (substring x)))))
(f (read-numeric) (read-encoded/unencoded))
\end{lstlisting}

There's a faux recursion to get the input into the right
format. Obviously this is contrived, but in a program that uses
polymorphic functions at different stages of its execution, the
temporal information helps pinpoint when and under what circumstances
an abstract address was poisoned with a value that causes an error.

\subsection{Precision on the frontier}
There is also a precision difference between the frontier-based method
and the state-space stepping semantics that is the initial model for a
widened store. That is, instead of separating states that need
stepping from seen states like the frontier-based method does, the
entire known state space steps with the current store:

\begin{align*}
\System &= \wp(\State) \times \Store \\
(S,\msto) &\machstep (S',\msto') \\
 \text{ where } I &= \setof{\mstate \mid \mastate \in S, \exists \msto'. \wn(\mastate,\msto) \machstep \mstate} \\
                S' &= \setof{\mastate \mid \exists \msto.\wn(\mastate,\msto) \in I} \\
                \msto' &= \bigsqcup\limits_{\msto \mid \exists\mastate.\wn(\mastate,\msto) \in I}{\msto}
\end{align*}
Although simple, this formulation has serious performance implications
and precision deficiencies compared to the frontier-based
semantics. This means the frontier-based algorithm is not just a more
efficient implementation strategy for a widened store semantics, its
results are not equivalent (they are better).

The reason for the difference is that store changes that would affect
the execution of some function might not be exercised before that
function may never be called after that change. The frontier semantics
would not end up stepping the state with the store dependency, whereas
the above semantics would.

A contrived example:
\begin{lstlisting}
(define b (box #f))
(define (foo x)
  (if (unbox b)
      (bar x)
      (bar (- x))))
(define (bar y) (/ (log y) (log 2))
(foo 20)
(set-box! b #t)
<code that doesn't invoke foo>
\end{lstlisting}

The frontier semantics would never find that {\tt bar} could cause an
exception for {\tt log} called on a non-positive number, whereas the
above semantics would, unnecessarily. Notice that not only are
bindings not poisoned, fewer paths are explored.

\end{document}
