\documentclass{llncs}
\usepackage{amsmath,amssymb,pfsteps,multicol,stmaryrd,mathpartir,centernot}
\newcommand{\superscript}[1]{\ensuremath{^{#1}}}
\newcommand{\subscript}[1]{\ensuremath{_{#1}}}
\newcommand{\tuple}[3][\ ]{{\tt #2{#1}}({#3})}

% Values
\newcommand{\clos}[1]{\tuple{clos}{#1}}
\newcommand{\rlos}[1]{\tuple{rlos}{#1}}

% States
\newcommand{\ev}[2][\ ]{\tuple[#1]{ev}{#2}}
\newcommand{\co}[1]{\tuple{co}{#1}}
\newcommand{\ap}[2][\ ]{\tuple[#1]{ap}{#2}}
\newcommand{\ans}[1]{\tuple{ans}{#1}}

% Continuations
\newcommand{\kmt}{\tt mt}
\newcommand{\kar}[2][\ ]{\tuple[#1]{ar}{#2}}
\newcommand{\kfn}[2][\ ]{\tuple[#1]{fn}{#2}}
\newcommand{\kif}[2][\ ]{\tuple[#1]{fi}{#2}}
\newcommand{\kuop}[2][\ ]{\tuple[#1]{oa}{#2}}
\newcommand{\kbopa}[2][\ ]{\tuple[#1]{oa1}{#2}}
\newcommand{\kbopb}[2][\ ]{\tuple[#1]{oa2}{#2}}

% Implementation forms
\newcommand{\generator}{{\tt generator}}
\newcommand{\yield}[1]{{\tt yield} #1}

% Syntax
\newcommand{\syntax}[1]{{\tt #1}}
\newcommand{\sapp}[3][\ ]{\tuple[#1]{app}{#2,#3}}
\newcommand{\slam}[3][\ ]{\tuple[#1]{lam}{#2,#3}}
\newcommand{\srec}[4][\ ]{\tuple[#1]{rec}{#2,#3,#4}}
\newcommand{\svar}[2][\ ]{\tuple[#1]{var}{#2}}
\newcommand{\snum}[2][\ ]{\tuple[#1]{num}{#2}}
\newcommand{\sbln}[2][\ ]{\tuple[#1]{bool}{#2}}
\newcommand{\sif}[4][\ ]{\tuple[#1]{if}{#2,#3,#4}}
\newcommand{\sop}[2][\ ]{\tuple[#1]{op}{#2}}
\newcommand{\sopu}[3][\ ]{\tuple[#1]{op}{#2,#3}}
\newcommand{\sopb}[4][\ ]{\tuple[#1]{op2}{#2,#3,#4}}
\newcommand{\strue}{{\tt tt}}
\newcommand{\sfalse}{{\tt ff}}
\newcommand{\saddone}{{\tt add1}}
\newcommand{\ssubone}{{\tt sub1}}
\newcommand{\szerohuh}{\syntax{zero?}}
\newcommand{\szero}{\syntax{0}}
\newcommand{\slit}[2][\ ]{\tuple[#1]{lit}{#2}}

\newcommand{\sNum}{\syntax{Z}}

% Metavariables
\newcommand{\maddr}{a}
\newcommand{\mvar}{x}
\newcommand{\mvarf}{f}
\newcommand{\mexp}{e}
\newcommand{\mexpi}[1]{e_{#1}}
\newcommand{\mexpf}{f}
\newcommand{\menv}{\rho}
\newcommand{\mkont}{\kappa}
\newcommand{\msto}{\sigma}
\newcommand{\mop}{o}
\newcommand{\mval}{v}
\newcommand{\mnum}{z}
\newcommand{\mbln}{b}
\newcommand{\mvalx}[1]{#1}
\newcommand\machstep{\longmapsto}
\newcommand\multimachstep{\longmapsto\!\!\!\!\!\rightarrow}
\newcommand{\mlit}{l}
\newcommand{\mstate}{\varsigma}
\newcommand{\mcomp}{k}
\newcommand{\mcompi}[1]{\mcomp_{#1}}
\newcommand{\interpdelta}{\Delta}

\newcommand{\compile}[1]{\llbracket#1\rrbracket}

\newcommand{\mlab}{{\ell}}
\newcommand{\mcntr}{{\delta}}
\newcommand{\mtcntr}{{\epsilon}}


\newcommand{\nequiv}{\centernot\equiv}
\newcommand{\finto}{\mathbin{\overset{\text{fin}}{\rightharpoonup}}}

\newcommand{\alloc}{\mathit{alloc}}
\newcommand{\dom}{\mathbf{dom}}
\newcommand{\inject}{\mathit{inject}}
\newcommand{\length}{\mathit{length}}
\newcommand{\cyclefree}{\mathit{cycle\text{-}free}}
\newcommand{\prop}{\mathit{prop}}
\newcommand{\lengthF}{{\mathcal F}}
\newcommand{\klength}{\mathit{klength}}
\newcommand{\possibletrace}{\mathit{possible\text{-}trace}}
\newcommand{\buildtable}{\mathit{build\text{-}table}}
\newcommand{\buildtablefrom}{\mathit{bt}}
\newcommand{\buildkont}{\mathit{build\text{-}kont}}
\newcommand{\reverse}{\mathit{reverse}}

\newcommand{\absN}{\hat{{\mathbb N}}}
\newcommand{\absn}{\hat{n}}
\newcommand{\squish}{\mathbin{\otimes}}

\newcommand{\many}[1]{\overline{#1}}

\newcommand{\var}{x}
\newcommand{\app}[2]{(#1\ #2)}
\newcommand{\lam}[2]{\lambda\ #1.\ #2}
\newcommand{\closure}[3]{(\lambda #1. #2,\ #3)}
\newcommand{\Var}{\mathit{Var}}

\newcommand{\expr}{e}

\newcommand{\addr}{a}

\newcommand{\ktab}{\Xi}
\newcommand{\ktabof}{\mathit{KTable\text{-}of}}
\newcommand{\Ktab}{\mathit{KTable}}
\newcommand{\Kkey}{\mathit{KTableKey}}
\newcommand{\ktabkey}[3]{(#1,\ #2,\ #3)}
\newcommand{\ktabentry}[2]{(#1,\ #2)}

\newcommand{\val}{v}

\newcommand{\kont}{\kappa}

\newcommand{\env}{\rho}

\newcommand{\sto}{\sigma}

\newcommand{\conf}{\varsigma}
\newcommand{\Conf}{\mathit{Conf}}

\newcommand{\setof}[1]{\{ #1 \}}

\newcommand{\redto}{\mathbin{\longmapsto}}
\newcommand{\redtraceto}[1][\ ]{\mathbin{\longmapsto^{TR}_{#1}}}

\newcommand{\alt}{\mid}
\newcommand{\mt}{\mathbf{mt}}
\newcommand{\ar}[3]{\mathbf{ar}(#1,\ #2,\ #3)}
\newcommand{\fn}[2]{\mathbf{fn}(#1,\ #2)}
\newcommand{\rt}[2]{\texttt{[]}^{(#1,\ #2)}}

\begin{document}

For reasons of space, the entire definitions of each intermediate
semantics were not shown (just the original and lazy, abstractly
compiled). We thus give a reference here to the complete semantics of
each other machine to use as a basis for the following proofs.

\section{Semantics}

\subsection{ISWIM with store-allocated results}

Machine configuration space:
\begin{align*}
\mstate \in \State &= \ev[^\mcntr]{\mexp, \menv, \msto, \mkont} \alt
                     \co{\mkont, \mval, \msto} \alt
                     \ap[_\mlab^\mcntr]{\mval,\mval,\msto,\mkont} \alt
                     \ans{\msto,\mval} \\
\mkont \in \Kont &::= \kmt \alt
                      \kar[_\mlab^\mcntr]{\mexp, \menv, \maddr} \alt
                      \kfn[_\mlab^\mcntr]{\maddr, \maddr} \alt
                      \kif[_\mlab^\mcntr]{\mexp, \mexp, \menv, \mkont} \\
\mlab \in \Label &\text{ an infinite set} \\
\mcntr \in \Counter &= \Label^* \\
\mval \in \Value &::= \mlit \alt \mop \alt
                      \clos{\mvar,\mexp,\menv} \\
\mstor \in \Storeable &::= \mlit \alt \mop \alt
                           \clos{\mvar,\mexp,\menv} \alt
                           \mkont \\
\menv \in \Env &= \Var \finto \Addr \\
\msto \in \Store &= \Addr \finto \wp(\Storeable)
\end{align*}

Reduction semantics:
\begin{gather*}
\begin{array}{@{}r@{\ }c@{\ }l@{}}
%% EVAL
\ev{\svar\mvar,\menv,\msto,\mkont} &\machstep&
\co{\mkont,\saddr{\menv(\mvar)},\msto}
\\
\ev{\slit\mlit,\menv,\msto,\mkont} &\machstep&
\co{\mkont,\mlit,\msto}
\\
\ev{\slam\mvar\mexp,\menv,\msto,\mkont} &\machstep&
\co{\mkont,\clos{\mvar,\mexp,\menv},\msto}
\\
\ev[^\mcntr]{\sapp[^\mlab]{\mexpi0}{\mexpi1},\menv,\msto,\mkont} &\machstep&
\ev[^\mcntr]{\mexpi{0},\menv,\msto',\kar[_\mlab^\mcntr]{\mexpi{1},\menv,\maddr}}
\\
&&
\text{ where }\maddr,\msto' = \push^\mcntr_\mlab(\msto,\mkont)
\\
\ev[^\mcntr]{\sif[^\mlab]{\mexpi0}{\mexpi1}{\mexpi2},\menv,\msto,\mkont} &\machstep&
\ev[^\mcntr]{\mexpi0,\menv,\msto',\kif[^\mcntr]{\mexpi1,\mexpi2,\menv,\maddr}}
\\
&&
\text{ where }\maddr,\msto' = \push_\mlab^\mcntr(\msto,\mkont)
\\[2mm]
%% CONTINUE
\co{\kmt,\mval,\msto} &\machstep&
\ans{\msto,\mval}
\\
\co{\kar[^\mcntr_\mlab]{\mexp,\menv,\maddr},\mval,\msto} & \machstep&
\ev[^\mcntr]{\mexp,\menv,\msto',\kfn[^\mcntr_\mlab]{\maddralt,\maddr}}
\\
&&
\text{ where }\maddralt,\msto' = \stash_\mlab^\mcntr(\msto,\mval)
\\
\co{\kfn[^\mcntr_\mlab]{\maddralt,\maddr},\mval,\msto} & \machstep&
\ap[^\mcntr_\mlab]{\mvalx{u},\mval,\mkont,\msto}
\text{ where } \mkont \in \msto(\maddr), \mvalx{u} \in \msto(\maddralt)
\\
\co{\kif[^\mcntr]{\mexpi0,\mexpi1,\menv,\maddr},\strue,\msto} & \machstep&
\ev[^\mcntr]{\mexpi0,\menv,\msto,\mkont}
\text{ where }\mkont\in\msto(\maddr)
\\
\co{\kif[^\mcntr]{\mexpi0,\mexpi1,\menv,\maddr},\sfalse,\msto} & \machstep&
\ev[^\mcntr]{\mexpi1,\menv,\msto,\mkont}
\text{ where }\mkont\in\msto(\maddr)
\\[2mm]
%% APPLY
\ap[^\mcntr_\mlab]{\clos{\mvar,\mexp,\menv},\mval,\msto,\mkont} & \machstep&
\ev[^{\mcntr'}\!]{\mexp,\menv',\msto',\mkont}
\\
\multicolumn{3}{r@{}}{
\text{ where }\menv',\msto',\mcntr' = \bind^{ \mcntr}_\mlab(\menv,\msto,\mvar,\mval)}
\\
\ap[^\mcntr_\mlab]{\mop,\mval,\msto,\mkont} & \machstep&
\co{\mkont,\mval',\msto}
\text{ where } \mval'\in\interpdelta(\mop,\mval)
\end{array}
\end{gather*}

\begin{align*}
\push_\mlab^\mcntr(\msto,\mstor) &= \maddr,\msto\sqcup[\maddr\mapsto\{\mkont\}]
\mbox{ where }\maddr \notin\msto
\\
\stash_\mlab^\mcntr(\msto,\mval) &= \maddr,\msto\sqcup[\maddr\mapsto\{\mval\}]
\mbox{ where }\maddr \notin\msto
\\
\bind_\mlab^\mcntr(\menv,\msto,\mvar,\mval) &= \menv[\mvar\mapsto\maddr],\msto\sqcup[\maddr\mapsto\{\mval\}],\lfloor \mlab\mcntr \rfloor_k
\mbox{ where }\maddr \notin\msto
\end{align*}

\begin{align*}
\inject(\mexp) &= \ev[^\mtcntr]{\mexp,\bot,\bot,\kmt} \\
\reachable{\mexp} &= \setof{\mstate \mid \inject(\mexp) \machstep^* \mstate}
\end{align*}

\subsection{Store-allocated results with lazy nondeterminism}

Machine configuration space:
\begin{align*}
\mstate \in \State &= \ev[^\mcntr]{\mexp, \menv, \msto, \mkont} \alt
                     \co{\mkont, \mval, \msto} \alt
                     \ap[_\mlab^\mcntr]{\mval,\mval,\msto,\mkont} \alt
                     \ans{\msto,\mval}\\
\mval \in \Value &::= \mlit \alt \mop \alt
                      \clos{\mvar,\mexp,\menv} \alt
                      \saddr\maddr
\end{align*}
$\Kont$, $\Storeable$, $\Env$ and $\Store$ are defined the same as previously.

Reduction semantics:
\begin{gather*}
\begin{array}{@{}r@{\ }c@{\ }l@{}}
%% EVAL
\ev{\svar\mvar,\menv,\msto,\mkont} &\machstep&
\co{\mkont,\saddr{\menv(\mvar)},\msto}
\\
\ev{\slit\mlit,\menv,\msto,\mkont} &\machstep&
\co{\mkont,\mlit,\msto}
\\
\ev{\slam\mvar\mexp,\menv,\msto,\mkont} &\machstep&
\co{\mkont,\clos{\mvar,\mexp,\menv},\msto}
\\
\ev[^\mcntr]{\sapp[^\mlab]{\mexpi0}{\mexpi1},\menv,\msto,\mkont} &\machstep&
\ev[^\mcntr]{\mexpi{0},\menv,\msto',\kar[_\mlab^\mcntr]{\mexpi{1},\menv,\maddr}}
\\
&&
\text{ where }\maddr,\msto' = \push^\mcntr_\mlab(\msto,\mkont)
\\
\ev[^\mcntr]{\sif[^\mlab]{\mexpi0}{\mexpi1}{\mexpi2},\menv,\msto,\mkont} &\machstep&
\ev[^\mcntr]{\mexpi0,\menv,\msto',\kif[^\mcntr]{\mexpi1,\mexpi2,\menv,\maddr}}
\\
&&
\text{ where }\maddr,\msto' = \push_\mlab^\mcntr(\msto,\mkont)
\\[2mm]
%% CONTINUE
\co{\kmt,\mval,\msto} &\machstep&
\ans{\msto,\mvalx{u}}
\text{ where } \mvalx{u} \in \force(\msto, \mval)
\\
\co{\kar[^\mcntr_\mlab]{\mexp,\menv,\maddr},\mval,\msto} & \machstep&
\ev[^\mcntr]{\mexp,\menv,\msto',\kfn[^\mcntr_\mlab]{\maddralt,\maddr}}
\\
&&
\text{ where }\maddralt,\msto' = \stash_\mlab^\mcntr(\msto,\mval)
\\
\co{\kfn[^\mcntr_\mlab]{\maddralt,\maddr},\mval,\msto} & \machstep&
\ap[^\mcntr_\mlab]{\mvalx{u},\mval,\mkont,\msto}
\text{ where } \mkont \in \msto(\maddr), \mvalx{u} \in \msto(\maddralt)
\\
\co{\kif[^\mcntr]{\mexpi0,\mexpi1,\menv,\maddr},\strue,\msto} & \machstep&
\ev[^\mcntr]{\mexpi0,\menv,\msto,\mkont}
\text{ where }\mkont\in\msto(\maddr)
\\
\co{\kif[^\mcntr]{\mexpi0,\mexpi1,\menv,\maddr},\sfalse,\msto} & \machstep&
\ev[^\mcntr]{\mexpi1,\menv,\msto,\mkont}
\text{ where }\mkont\in\msto(\maddr)
\\[2mm]
%% APPLY
\ap[^\mcntr_\mlab]{\clos{\mvar,\mexp,\menv},\mval,\msto,\mkont} & \machstep&
\ev[^{\mcntr'}\!]{\mexp,\menv',\msto',\mkont}
\\
\multicolumn{3}{r@{}}{
\text{ where }\menv',\msto',\mcntr' = \bind^{ \mcntr}_\mlab(\menv,\msto,\mvar,\mval)}
\\
\ap[^\mcntr_\mlab]{\mop,\mval,\msto,\mkont} & \machstep&
\co{\mkont,\mval',\msto}
\text{ where } \mvalx{u} \in \force(\msto, \mval), \mval'\in\interpdelta(\mop,\mvalx{u})
\end{array}
\end{gather*}

\begin{align*}
\force(\msto,\saddr\maddr) &= \msto(\maddr) \\
\force(\msto,\mval) &= \{ \mval \} \\
\stash(\msto,\mval) &= \maddralt, \msto\sqcup[\maddralt \mapsto \force(\msto, \mval)]
\mbox{ where }\maddr \notin\msto \\
\bind_\mlab^\mcntr(\menv,\msto,\mvar,\mval) &= \menv[\mvar\mapsto\maddr],\msto\sqcup[\maddr\mapsto\force(\msto,\mval)],\lfloor \mlab\mcntr \rfloor_k
\mbox{ where }\maddr \notin\msto \\
\push_\mlab^\mcntr(\msto,\mstor) &= \maddr,\msto\sqcup[\maddr\mapsto\{\mkont\}]
\mbox{ where }\maddr \notin\msto
\end{align*}

\begin{align*}
\inject(\mexp) &= \ev[^\mtcntr]{\mexp,\bot,\bot,\kmt} \\
\reachable{\mexp} &= \setof{\mstate \mid \inject(\mexp) \machstep^* \mstate}
\end{align*}

\subsection{Lazy nondeterminism with abstract compilation}

Machine configuration space:
\begin{align*}
\mstate \in \State &= \co{\mkont, \mval, \msto} \alt
                     \ap[_\mlab^\mcntr]{\mval,\mval,\msto,\mkont} \alt
                     \ans{\msto, \mval}\\
\mcomp \in \Compiled &= (\Env \times \Store \times \Kont \times \Counter) \to \State \\
\mkont \in \Kont &::= \kmt \alt
                      \kar[_\mlab^\mcntr]{\mcomp, \menv, \maddr} \alt
                      \kfn[_\mlab^\mcntr]{\maddr, \maddr} \alt
                      \kif[_\mlab^\mcntr]{\mcomp, \mcomp, \menv, \maddr} \\
\mval \in \Value &::= \mlit \alt \mop \alt
                      \clos{\mvar,\mcomp,\menv} \alt
                      \saddr\maddr \\
\Storeable &::= \mlit \alt \mop \alt
                \clos{\mvar,\mcomp,\menv} \alt
                \mkont \\
\end{align*}
$\Store$ and $\Env$ are defined the same as previously.

We write $\lambda^\mcntr(\mathit{args}\ldots). \mathit{body}$ (and without superscript) to mean
$\lambda(\mathit{args}\ldots \mcntr). \mathit{body}$ and
$\mcomp^\mcntr(\menv, \msto, \mkont)$ to mean $\mcomp(\menv, \msto,
\mkont, \mcntr)$ for notational consistency.

Abstract compilation function:

\begin{align*}
\compile{\_} &: \Expr \to \Compiled \\
\compile{\svar\mvar} &= \lambda(\menv,\msto,\mkont) .\co{\mkont,\saddr{\menv(\mvar)},\msto}
\\
\compile{\slit\mlit} &= \lambda(\menv,\msto,\mkont) .
\co{\mkont,\mlit,\msto}
\\
\compile{\slam\mvar\mexp} &= \lambda(\menv,\msto,\mkont) .
\co{\mkont,\clos{\mvar,\compile\mexp,\menv},\msto}
\\
\compile{\sapp[^\mlab]{\mexpi0}{\mexpi1}} &= \lambda^\mcntr (\menv,\msto,\mkont) .
\compile{\mexpi0}^\mcntr(\menv,\msto',\kar[_\mlab^\mcntr]{\compile{\mexpi1},\menv,\maddr})
\\
&
\text{ where }\maddr,\msto' = \push^\mcntr_\mlab(\msto,\mkont)
\\
\compile{\sif[^\mlab]{\mexpi0}{\mexpi1}{\mexpi2}} &= \lambda^\mcntr(\menv,\msto,\mkont) .
\compile{\mexpi0}^\delta(\menv,\msto',\kif[^\mcntr]{\compile{\mexpi1},\compile{\mexpi2},\menv,\maddr})
\\
&\text{ where }\maddr,\msto' = \push_\mlab^\mcntr(\msto,\mkont)
\end{align*}

Reduction semantics:

\begin{gather*}
\begin{align*}
%% CONTINUE
\co{\kmt,\mval,\msto} &\machstep
\ans{\msto,\mvalx{u}}
\text{ where } \mvalx{u} \in \force(\msto,\mval)
\\
\co{\kar[^\mcntr_\mlab]{\mcomp,\menv,\maddr},\mval,\msto} & \machstep
\mcomp^\mcntr(\menv,\msto',\kfn[^\mcntr_\mlab]{\maddralt,\maddr})
\\
\text{ where }\maddralt,\msto' &= \stash_\mlab^\mcntr(\msto,\mval)
\\
\co{\kfn[^\mcntr_\mlab]{\maddralt,\maddr},\mval,\msto} & \machstep
\ap[^\mcntr_\mlab]{\mvalx{u},\mval,\msto,\mkont}
\text{ where } \mkont \in \msto(\maddr), \mvalx{u} \in \msto(\maddralt)
\\
\co{\kif[^\mcntr]{\mcompi0,\mcompi1,\menv,\maddr},\strue,\msto} & \machstep
\mcompi0^\mcntr(\menv,\msto,\mkont)
\text{ where }\mkont\in\msto(\maddr)
\\
\co{\kif[^\mcntr]{\mcompi0,\mcompi1,\menv,\maddr},\sfalse,\msto} & \machstep
\mcompi1^\mcntr(\menv,\msto,\mkont)
\text{ where }\mkont\in\msto(\maddr)
\\[2mm]
%% APPLY
\ap[^\mcntr_\mlab]{\clos{\mvar,\mcomp,\menv},\mval,\msto,\mkont} & \machstep
\mcomp^{\mcntr'}(\menv',\msto',\mkont) \\
\text{ where }\menv',\msto',\mcntr' &= \bind^\mcntr_\mlab(\menv,\msto,\mvar,\mval)
\\
\ap{\mop,\mval,\msto,\mkont} & \machstep
\co{\mkont,\mval',\msto} \\
\text{ where }\mkont &\in\msto(\maddr)
\text{ and } \mvalx{u} \in \force(\msto,\mval), \mval'\in\interpdelta(\mop,\mvalx{u})
\end{align*}
\end{gather*}

\begin{align*}
\inject(\mexp) &= \compile{\mexp}^{\mtcntr}(\bot,\bot,\kmt) \\
\reachable{\mexp} &= \setof{\mstate \mid \inject(\mexp) \machstep^* \mstate}
\end{align*}

\subsection{Widened abstract compilation}

Machine configuration space:
\begin{align*}
\mstate \in \State &= \co{\mkont, \mval} \alt
                     \ap[_\mlab^\mcntr]{\mval,\mval,\mkont} \alt
                     \ans{\mval} \\
\System &= \wp(\State) \times \Store
\end{align*}

\begin{align*}
\nw(\co{\mkont,\mval,\msto}) &= \co{\mkont, \mval}, \msto \\
\nw(\ap[^\mcntr_\mlab]{\mvalx{u},\mval,\msto,\mkont}) &= \ap[^\mcntr_\mlab]{\mvalx{u},\mval,\mkont}, \msto \\
\nw(\ans{\msto,\mval}) &= \ans{\mval}, \msto \\
\wn(\co{\mkont,\mval},\msto) &= \co{\mkont,\mval,\msto} \\
\wn(\ap[^\mcntr_\mlab]{\mvalx{u},\mval,\mkont},\msto) &= \ap[^\mcntr_\mlab]{\mvalx{u},\mval,\msto,\mkont} \\
\wn(\ans{\mval},\msto) &= \ans{\msto,\mval}
\end{align*}

Reduction semantics:
\begin{align*}
\inject(\mexp) &= (\setof{\ttuple{\mstate'}{\msto}},\setof{\mstate'},\msto) \\
 \text{ where } \mstate &= \compile{\mexp}^{\mtcntr}(\bot,\bot,\kmt) \\
                \mstate', \msto &= \nw(\mstate) \\
\reachable{\mexp} &= \setof{\wn(\mstate,\msto') \mid
                             \inject(\mexp) \machstep^* (S,F,\msto),
                             \ttuple{\mstate}{\msto'} \in S} \\
(S,F,\msto) &\machstep (S \cup S', F', \msto') \\
 \text{ where }
  I &= \setof{ \nw(\mstate^*) \mid \mstate \in F, \wn(\mstate,\msto) \machstep \mstate^*, \nw(\mstate^*)\notin S} \\
  F' &= \setof{\mstate \mid \exists \msto. \ttuple{\mstate}{\msto} \in S'} \\
  \Sigma &= \setof{\msto \mid \exists \mstate. \ttuple{\mstate}{\msto} \in S'} \\
  \msto' &= \bigsqcup\limits_{\msto\in\Sigma}{\msto} \\
  S' &= \setof{\ttuple{\mstate}{\msto'} \mid \mstate \in F'}
\end{align*}

\subsection{Abstract compilation with store deltas}

All previous machines had a trivial widening operator for the store
that would expand states without stores to states with stores, reduce
with the written semantics, and then remove the resulting stores and
join them again so that there is one store shared amongst all
states. Here we have a different widening that accumulates store
changes so that entire stores need not be joined each step - just
their changes.
\\
Machine configuration space:
\begin{align*}
%\System &= \wp(\State) \times \Store \\
\mstate \in \State &= \co{\mkont, \mval} \alt
                     \ap[_\mlab^\mcntr]{\mval,\mval,\mkont} \alt
                     \ans{\mval} \\
\mcomp \in \Compiled &=
  (\Env \times \Store \times \StoreDelta \times \Kont \times \Counter) \to
   (\State \times \StoreDelta) \\
\mkont \in \Kont &::= \kmt \alt
                      \kar[_\mlab^\mcntr]{\mcomp, \menv, \maddr} \alt
                      \kfn[_\mlab^\mcntr]{\maddr, \maddr} \alt
                      \kif[_\mlab^\mcntr]{\mcomp, \mcomp, \menv, \maddr} \\
\msdiff \in \StoreDelta &= \Addr \finto \wp(\Storeable) \\
\end{align*}
$\Storeable$, $\Store$, $\Env$ and $\Value$ are defined the same as previously.

Abstract compilation function:

\begin{align*}
\compile{\_} &: \Expr \to \Compiled \\
\compile{\svar\mvar} &= \lambda(\menv,\msto,\msdiff, \mkont) .\ttuple{\co{\mkont,\saddr{\menv(\mvar)},\msto}}{\bot}
\\
\compile{\slit\mlit} &= \lambda(\menv,\msto,\msdiff,\mkont) .
\ttuple{\co{\mkont,\mlit,\msto}}{\bot}
\\
\compile{\slam\mvar\mexp} &= \lambda(\menv,\msto,\msdiff,\mkont) .
\ttuple{\co{\mkont,\clos{\mvar,\compile\mexp,\menv},\msto}}{\bot}
\\
\compile{\sapp[^\mlab]{\mexpi0}{\mexpi1}} &= \lambda^\mcntr (\menv,\msto,\msdiff,\mkont) .
\compile{\mexpi0}^\mcntr(\menv,\msto,\msdiff',\kar[_\mlab^\mcntr]{\compile{\mexpi1},\menv,\maddr})
\\
&
\text{ where }\maddr,\msdiff' = \push^\mcntr_\mlab(\msdiff,\mkont)
\\
\compile{\sif[^\mlab]{\mexpi0}{\mexpi1}{\mexpi2}} &= \lambda^\mcntr(\menv,\msto,\msdiff,\mkont) .
\compile{\mexpi0}^\mcntr(\menv,\msto,\msdiff',\kif[^\mcntr]{\compile{\mexpi1},\compile{\mexpi2},\menv,\maddr})
\\
&\text{ where }\maddr,\msdiff' = \push_\mlab^\mcntr(\msdiff,\mkont)
\end{align*}

Reduction semantics helper (write $\mstate \machstep_\msto \mstate',\msdiff$ to mean $((\mstate,\msto),(\mstate',\msdiff)) \in \machstep$):

\begin{gather*}
\begin{align*}
\machstep &\subseteq (\State \times \Store) \times (\State \times \StoreDelta) \\
%% CONTINUE
\co{\kmt,\mval} &\machstep_\msto
\ans{\mvalx{u}},\bot
\text{ where } \mvalx{u} \in \force(\msto,\mval)
\\
\co{\kar[^\mcntr_\mlab]{\mcomp,\menv,\maddr},\mval} & \machstep_\msto
\mcomp^\mcntr(\menv,\kfn[^\mcntr_\mlab]{\maddralt,\maddr}), \msdiff
\\
\text{ where }\maddralt,\msdiff &= \stash_\mlab^\mcntr(\msto,\mval)
\\
\co{\kfn[^\mcntr_\mlab]{\maddralt,\maddr},\mval} & \machstep_\msto
\ap[^\mcntr_\mlab]{\mvalx{u},\mval,\mkont}, \bot
\text{ where } \mkont \in \msto(\maddr), \mvalx{u} \in \msto(\maddralt)
\\
\co{\kif[^\mcntr]{\mcompi0,\mcompi1,\menv,\maddr},\strue} & \machstep_\msto
\mcompi0^\mcntr(\menv,\msto,\bot,\mkont), \bot
\text{ where }\mkont\in\msto(\maddr)
\\
\co{\kif[^\mcntr]{\mcompi0,\mcompi1,\menv,\maddr},\sfalse} & \machstep_\msto
\mcompi1^\mcntr(\menv,\msto,\bot,\mkont), \bot
\text{ where }\mkont\in\msto(\maddr)
\\[2mm]
%% APPLY
\ap[^\mcntr_\mlab]{\clos{\mvar,\mcomp,\menv},\mval,\mkont} & \machstep_\msto
\mcomp^{\mcntr'}(\menv',\msto,\msdiff,\mkont), \msdiff \\
\text{ where }\menv',\msdiff,\mcntr' &= \bind^\mcntr_\mlab(\menv,\msto,\mvar,\mval)
\\
\ap{\mop,\mval,\mkont} & \machstep
\co{\mkont,\mval'},\bot \\
\text{ where }\mkont &\in\msto(\maddr)
\text{ and } \mvalx{u} \in \force(\msto,\mval), \mval'\in\interpdelta(\mop,\mvalx{u})
\end{align*}
\end{gather*}

Reduction semantics:
\begin{align*}
\inject(\mexp) &= (\setof{\ttuple{\mstate}{\msto}}, \setof{\mstate}, \msto) \\
 \text{ where } \mstate, \msdiff &= \compile{e}^{\mtcntr}(\bot,\bot,\bot,\kmt) \\
                \msto &= \replay(\msdiff,\bot) \\
\reachable{\mexp} &= \setof{\wn(\mstate,\msto') \mid
                             \inject(\mexp) \machstep^* (S,F,\msto),
                             \ttuple{\mstate}{\msto'} \in S} \\
(S,F,\msto) &\machstep (S \cup S', F', \msto') \\
 \text{ where } I &=
   \setof{ \ttuple{\mstate'}{\msdiff} \mid \mstate \in F, \mstate \machstep_\msto \mstate', \msdiff} \\
   F' &= \setof{\mstate \mid \exists \msdiff. \ttuple{\mstate}{\msdiff} \in I} \\
   \Xi &= \setof{\msdiff \mid \exists \mstate. \ttuple{\mstate}{\msdiff} \in I} \\
   \msto' &= \replayall(\Xi,\msto) \\
   S' &= \setof{\ttuple{\mstate}{\msto'} \mid \mstate \in F'} \\
   \replayall(\varnothing,\msto) &= \msto \\
   \replayall(\setof{\msdiff} \cup \Xi, \msto) &= \replayall(\Xi, \replay(\msdiff, \msto))
\end{align*}

%% This reduction semantics has an expensive fixed point computation
%% since many states would be stepped several times, even if
%% unnecessary. Instead, we use a frontier solution that maps states to
%% the last store they were processed at to determine if they need to be
%% stepped again.

%% Frontier computation:
%% \begin{align*}
%% \mathit{reachable}(e) &= \lfp{{\mathcal F}}(\varnothing, \setof{\mstate}, \replay(\msdiff,\bot)) \\
%%  \text{ where } \mstate, \msdiff &= \compile{e}^\mcntr(\bot,\bot,\bot,\kmt) \\
%% {\mathcal F} &: ((\State \finto \Store) \times \wp(\State) \times \Store) \to
%%                 ((\State \finto \Store) \times \wp(\State) \times \Store) \\
%% {\mathcal F}(S,F,\msto) &= (S', F \cup F', \msto') \\
%%  \text{ where }
%%   I &= \setof{ \ttuple{\mstate'}{\msdiff} \mid \mstate \in F, \mstate \machstep_\msto \mstate', \msdiff} \\
%%   \msto' &= \replayall(\setof{\msdiff \mid \exists \mstate. (\mstate, \msdiff) \in I}, \msto) \\
%%   F' &= \setof{\mstate \mid \exists \msdiff. (\mstate,\msdiff) \in I,
%%                 \msto' \neq S(\mstate)} \\
%%   S' &= \lambda \mstate. \left\{\begin{array}{ll}
%%                                  \msto' & \text{if } \mstate \in F' \\
%%                                  S(\mstate) & \text{otherwise}
%%                                 \end{array}\right.
%% \end{align*}

\subsection{Store deltas with timestamped store}

\begin{align*}
\inject(\mexp) &=
 (S_0, \setof{\mstate},
  \replay(\msdiff,\bot), 0) \\
 \text{ where } \mstate, \msdiff &= \compile{e}^\mcntr(\bot,\bot,\bot,\kmt) \\
                S_0 &= \lambda\mstate'.
                        \left\{\begin{array}{ll}
                         0 & \text{if } \mstate' = \mstate \\
                         \epsilon & \text{otherwise}
                        \end{array}\right. \\
\mathit{reachable}(\mexp) &=
  \setof{\wn(\mstate,\Sigma(n)) \mid
         \inject(\mexp) \machstep^* (S,F,\Sigma,n'), \hd(S(\mstate)) = n}
\end{align*}

\begin{align*}
\System &= (\State \to {\mathbb N}^*) \times \wp(\State) \times ({\mathbb N} \finto \Store) \times {\mathbb N} \\
(S,F,\Sigma,n) &\machstep (S', F \cup F', \Sigma[n' \mapsto \msto'],n') \\
 \text{ where }
 \msto &= \Sigma(n) \\
  I &= \setof{ \ttuple{\mstate'}{\msdiff} \mid \mstate \in F, \mstate \machstep_\msto \mstate', \msdiff} \\
  \msto',\updatedp &=
     \replaychangeall(\setof{\msdiff \mid \exists \mstate. (\mstate, \msdiff) \in I}, \msto,\sfalse) \\
  n' &= \left\{\begin{array}{ll}
                n + 1 & \text{if } \updatedp \\
                n & \text{otherwise}
               \end{array}\right. \\
  F' &= \setof{\mstate \mid \exists \msdiff. (\mstate,\msdiff) \in I,
                n' \neq \hd(S(\mstate))} \\
  S' &= \lambda \mstate. \left\{\begin{array}{ll}
                                 n' S(\mstate) & \text{if } \mstate \in F' \\
                                 S(\mstate) & \text{otherwise}
                                \end{array}\right.
\end{align*}

\begin{align*}
  \replaychangeall(\setof{\msdiff} \cup \Xi, \msto, \updatedp) &=
    \replaychangeall(\Xi, \msto', \updatedp \vee \joinp) \\
   \text{ where } \msto', \joinp &= \replaychange(\msdiff, \msto, \sfalse) \\
  \replaychangeall(\varnothing, \msto, \updatedp) &= \msto, \updatedp \\
  \replaychange(\msdiff[\maddr \mapsto S], \msto, \joinp) &=
   \replaychange(\msdiff, \msto', \joinp \vee \joinp') \\
  \text{ where } S' &= S \sqcup \msto(\maddr) \\
                 \joinp' &= \msto(\maddr) \deceq S' \\
                 \msto' &= \msto[\maddr \mapsto S'] \\
  \replaychange(\bot, \msto, \joinp) &= \msto, \joinp
\end{align*}

\section{Proofs}

\subsection{Soundness of lazy-nondeterminism}
Soundness is apparent when we consider $\saddr\maddr$ to represent all
values in $\msto(\maddr)$. We make the argument simpler by requiring
the order on stores to only relate functions with the same domain. We
further require that fresh choice of addresses is deterministic based
on map domain. This keeps machinery like partial bijections and
renaming from cluttering the argument.

\begin{align*}
\alpha(\mstate) &= \mstate \\
\gamma(\ev[^\mcntr]{\mexp, \menv, \msto, \mkont}) &= \setof{\ev[^\mcntr]{\mexp, \menv, \msto, \mkont}} \\
\gamma(\co{\mkont, \mval, \msto}) &= \setof{\co{\mkont, \mval', \msto} \mid \mval' \in \force(\msto,\mval)} \\
\gamma(\ap[_\mlab^\mcntr]{\mval, \mvalx{u}, \msto}) &= \setof{\ap[_\mlab^\mcntr]{\mval', \mvalx{u}', \msto} \mid \mval' \in \force(\msto,\mval), \mvalx{u}' \in \force(\msto,\mvalx{u})} \\
\gamma(\ans{\msto, \mval}) &= \setof{\ans{\msto, \mval'} \mid \mval' \in \force(\msto,\mval)} \\
\alpha^*(C) &= \setof{\alpha(\mstate) \mid \mstate \in C} \\
\gamma^*(A) &= \bigcup\limits_{\hat{\mstate} \in A}{\gamma(\hat{\mstate})}
\end{align*}

\begin{mathpar}
\inferrule{\dom(\msto) = \dom(\msto') \\ \forall \maddr. \msto(\addr) \subseteq \msto'(\maddr)}
          {\msto \sqsubseteq \msto'} \\
\inferrule{\msto \sqsubseteq \msto'}
          {\ev[^\mcntr]{\mexp, \menv, \msto, \mkont} \sqsubseteq
           \ev[^\mcntr]{\mexp, \menv, \msto', \mkont}}
 \qquad
\inferrule{\force(\msto, \mval) \subseteq \force(\msto', \mval') \\ \msto \sqsubseteq \msto'}
          {\co{\mkont,\mval,\msto} \sqsubseteq \co{\mkont, \mval', \msto'}} \\
%
\inferrule{\force(\msto, \mval) \subseteq \force(\msto', \mval') \\
           \force(\msto, \mvalx{u}) \subseteq \force(\msto', \mvalx{u}') \\
           \msto \sqsubseteq \msto'}
{\ap[_\mlab^\mcntr]{\mval, \mvalx{u}, \msto} \sqsubseteq \ap[_\mlab^\mcntr]{\mval', \mvalx{u}', \msto'}} \\
%
\inferrule{\force(\msto, \mval) \subseteq \force(\msto', \mval') \\ \msto \sqsubseteq \msto'}
          {\ans{\msto, \mval} \sqsubseteq \ans{\msto', \mval'}}
 \qquad
\inferrule{\forall \hat{\mstate} \in S, \exists \hat{\mstate}' \in S'. \hat{\mstate} \sqsubseteq \hat{\mstate}'}
          {S \sqsubseteq S'}
\end{mathpar}

By definitions of $\alpha, \gamma, \force$, $\gamma^* \circ \alpha^* =
1_C$. Also $\alpha^*\circ\gamma^* \le 1_A$ is straightforward to prove. This forms a Galois connection.

\begin{theorem}If $\mstate \machstep \mstate'$ and $\alpha(\mstate) \sqsubseteq \hat{\mstate}$ then
$\exists \hat{\mstate}'. \hat{\mstate} \machstep \hat{\mstate}'$.
\end{theorem}
\begin{proof}
By assumption, $\gamma^*(\alpha^*(\setof{\mstate})) \sqsubseteq
\gamma^*{\setof{\hat{\mstate}}}$. By the above property, and
definition of $\gamma^*$, $\mstate \in \gamma(\hat{\mstate})$.
Since $\alpha$ does not introduce $\saddr{}$ values, most cases follow by definition.

By cases on $\mstate \machstep \mstate'$:
\begin{byCases}
%% EVAL
\case{\ev{\svar\mvar,\menv,\msto,\mkont} \machstep \co{\mkont,\mval,\msto}}{
 where $\mval \in \msto(\menv(\mvar))$

By assumption, $\hat{\mstate} \equiv \ev{\svar\mvar, \menv, \msto', \mkont}$ such that $\msto \sqsubseteq \msto'$.
\begin{pfsteps*}
 \item{Let $\hat{\mstate}' = \co{\mkont,\saddr{\menv(\mvar)},\msto'}$}
 \item{$\setof{\mstate'} \sqsubseteq \gamma(\hat{\mstate}')$} \BY{def. $\gamma$}
 \item{$\alpha(\mstate') \sqsubseteq \hat{\mstate}'$} \BY{def. $\alpha$, $\alpha^*\circ\gamma^* \le 1_A$}
\end{pfsteps*}}

\case{\ev{\slit\mlit,\menv,\msto,\mkont} \machstep \co{\mkont,\mlit,\msto}}{
By assumption, $\hat{\mstate} \equiv \ev{\svar\mvar, \menv, \msto', \mkont}$ such that $\msto \sqsubseteq \msto'$.
Let $\hat{\mstate}' = \co{\mkont,\mlit,\msto'}$. Conclusion holds by definition of
$\alpha, \sqsubseteq, \machstep$.}

\case{\ev{\slam\mvar\mexp,\menv,\msto,\mkont} \machstep \co{\mkont,\clos{\mvar,\mexp,\menv},\msto}}{
By assumption, $\hat{\mstate} \equiv \ev{\svar\mvar, \menv, \msto', \mkont}$ such that $\msto \sqsubseteq \msto'$.
Let $\hat{\mstate}' = \co{\mkont,\clos{\mvar,\mexp,\menv},\msto'}$. Conclusion holds by definition of
$\alpha, \sqsubseteq, \machstep$.}

\case{
\ev[^\mcntr]{\sapp[^\mlab]{\mexpi0}{\mexpi1},\menv,\msto,\mkont} \machstep
\ev[^\mcntr]{\mexpi{0},\menv,\msto',\kar[_\mlab^\mcntr]{\mexpi{1},\menv,\maddr}}}{
 where $\maddr,\msto' = \push^\mcntr_\mlab(\msto,\mkont)$

By assumption, $\hat{\mstate} \equiv \ev{\svar\mvar, \menv, \msto^*, \mkont}$ such that $\msto \sqsubseteq \msto^*$.
Let $\hat{\mstate}' = \ev[^\mcntr]{\mexpi{0},\menv,\msto^*_1,\kar[_\mlab^\mcntr]{\mexpi{1},\menv,\maddr}}$ where
$\maddr, \msto^*_1 = \push^\mcntr_\mlab(\msto^*,\mkont)$. Conclusion holds by definition of
$\alpha, \sqsubseteq, \machstep, \push$.}

\case{
\ev[^\mcntr]{\sif[^\mlab]{\mexpi0}{\mexpi1}{\mexpi2},\menv,\msto,\mkont} \machstep
\ev[^\mcntr]{\mexpi0,\menv,\msto',\kif[^\mcntr]{\mexpi1,\mexpi2,\menv,\maddr}}}{
 where $\maddr,\msto' = \push_\mlab^\mcntr(\msto,\mkont)$

By assumption, $\hat{\mstate} \equiv \ev{\sif[^\mlab]{\mexpi0}{\mexpi1}{\mexpi2},\menv,\msto^*,\kif[^\mcntr]{\mexpi1,\mexpi2,\menv,\maddr}}$ such that $\msto \sqsubseteq \msto^*$.
Let $\hat{\mstate}' = \ev[^\mcntr]{\mexpi0,\menv,\msto^*_1,\kif[^\mcntr]{\mexpi1,\mexpi2,\menv,\maddr}}$
where $\maddr,\msto^*_1 = \push_\mlab^\mcntr(\msto^*,\mkont)$.
Conclusion holds by definition of $\alpha, \sqsubseteq, \machstep, \push$.}

%% CONTINUE
\case{\co{\kmt,\mval,\msto} \machstep \ans{\msto,\mval}}{
By assumption, $\hat{\mstate} \equiv \co{\kmt,\mval',\msto^*}$ where $\msto \sqsubseteq \msto^*$ and $\force(\msto,\mval) \subseteq \force(\msto^*,\mval')$.

By definition of $\force, \sqsubseteq$,  $\mval \in \force(\msto^*,\mval')$.
Let $\hat{\mstate}' = \ans{\msto^*,\mval}$. Conclusion holds by definition of $\machstep, \force$.}

\case{
\co{\kar[^\mcntr_\mlab]{\mexp,\menv,\maddr},\mval,\msto} \machstep
\ev[^\mcntr]{\mexp,\menv,\msto',\kfn[^\mcntr_\mlab]{\maddralt,\maddr}}}{
 where $\maddralt,\msto' = \stash_\mlab^\mcntr(\msto,\mval)$

\begin{byCases}
\case{\hat{\mstate} \equiv \co{\kar[^\mcntr_\mlab]{\mexp,\menv,\maddr}, \saddr{\maddrx{c}}, \msto^*}}{
where $\msto \sqsubseteq \msto^*$

Let $\hat{\mstate}' = \ev[^\mcntr]{\mexp,\menv,\msto^*, \kfn[^\mcntr_\mlab]{\maddralt,\maddr}}$ where $\maddralt, \msto* = \stash(\msto,\saddr{\maddrx{c}})$. By definition of $\stash$,
 Conclusion holds by definition of $\alpha, \sqsubseteq, \stash, \machstep$.}
\otherwise{$\hat{\mstate} \equiv \co{\kar[^\mcntr_\mlab]{\mexp,\menv,\maddr}, \mval, \msto^*}$
 where $\msto \sqsubseteq \msto^*$ and $\mval \nequiv \saddr{\maddrx{c}}$.

Let $\hat{\mstate}' = \ev[^\mcntr]{\mexp,\menv,\msto^*_1,\kfn[^\mcntr_\mlab]{\maddralt,\maddr}}$
where $\maddralt, \msto^*_1 = \stash_\mlab^\mcntr(\msto^*,\mval)$.
Conclusion holds by definition of $\alpha, \sqsubseteq, \stash, \machstep$.}
\end{byCases}}

\case{
\co{\kfn[^\mcntr_\mlab]{\maddralt,\maddr},\mval,\msto} \machstep
\ap[^\mcntr_\mlab]{\mvalx{u},\mval,\mkont,\msto}}{
 where $\mkont \in \msto(\maddr), \mvalx{u} \in \msto(\maddralt)$

By assumption, $\hat{\mstate} \equiv \co{\kfn[^\mcntr_\mlab]{\maddralt,\maddr},\mval',\msto^*}$ where $\msto \sqsubseteq \msto^*$ and $\force(\msto,\mval) \subseteq \force(\msto^*,\mval')$.

By definition of $\sqsubseteq$, $\force$, $\mkont \in \msto^*(\maddr)$ and $\mvalx{u} \in \msto^*(\maddralt)$. Thus
letting $\hat{\mstate}' = \ap[^\mcntr_\mlab]{\mvalx{u},\mval,\mkont,\msto^*}$. Conclusion holds by definition of $\alpha, \machstep$.}

\case{
\co{\kif[^\mcntr]{\mexpi0,\mexpi1,\menv,\maddr},\strue,\msto} \machstep
\ev[^\mcntr]{\mexpi0,\menv,\msto,\mkont}}{
 where $\mkont\in\msto(\maddr)$

By assumption, $\hat{\mstate} \equiv
\co{\kif[^\mcntr]{\mexpi0,\mexpi1,\menv,\maddr},\strue,\msto^*}$ where
$\msto \sqsubseteq \msto^*$.

Let $\hat{\mstate}' = \ev[^\mcntr]{\mexpi0,\menv,\msto^*,\mkont}$. Conclusion holds by definition of $\alpha, \machstep$.}

\case{
\co{\kif[^\mcntr]{\mexpi0,\mexpi1,\menv,\maddr},\sfalse,\msto} \machstep
\ev[^\mcntr]{\mexpi1,\menv,\msto,\mkont}}{
 where $\mkont\in\msto(\maddr)$

By assumption, $\hat{\mstate} \equiv
\co{\kif[^\mcntr]{\mexpi0,\mexpi1,\menv,\maddr},\sfalse,\msto^*}$ where
$\msto \sqsubseteq \msto^*$.

Let $\hat{\mstate}' = \ev[^\mcntr]{\mexpi1,\menv,\msto^*,\mkont}$. Conclusion holds by definition of $\alpha, \machstep$.}

%% APPLY
\case{
\ap[^\mcntr_\mlab]{\clos{\mvar,\mexp,\menv},\mval,\msto,\mkont} \machstep
\ev[^{\mcntr'}\!]{\mexp,\menv',\msto',\mkont}}{
 where $\menv',\msto',\mcntr' = \bind^{ \mcntr}_\mlab(\menv,\msto,\mvar,\mval)$

By assumption, $\hat{\mstate} \equiv
\ap[^\mcntr_\mlab]{\clos{\mvar,\mexp,\menv},\mval',\msto^*,\mkont}$
where $\msto \sqsubseteq \msto^*$ and $\force(\msto,\mval) \subseteq
\force(\msto^*,\mval')$.

Let $\hat{\mstate}' =
\ev[^{\mcntr'}\!]{\mexp,\menv',\msto^*_1,\mkont}$ where
$\menv',\msto^*_1,\mcntr' =
\bind^\mcntr_\mlab(\menv,\msto^*,\mvar,\mval')$. Conclusion holds by
definition of $\alpha, \bind, \sqsubseteq, \machstep$.}

\case{
\ap[^\mcntr_\mlab]{\mop,\mval,\msto,\mkont} \machstep
\co{\mkont,\mval',\msto}}{
 where $\mval'\in\interpdelta(\mop,\mval)$

By assumption, $\hat{\mstate} \equiv
\ap[^\mcntr_\mlab]{\mop,\mval^*,\msto^*,\mkont}$ where $\msto \sqsubseteq \msto^*$ and $\force(\msto,\mval) \subseteq \force(\msto^*,\mval^*)$. By definition of $\force$, $\mval \in \force(\msto^*,\mval')$. 

Let $\hat{\mstate}' = \co{\mkont, \mval',\msto^*}$. Conclusion holds by definition of $\alpha, \machstep$.}
\end{byCases}
\end{proof}

\subsection{Semantic equivalence with abstract compilation}

We show that in the presence of abstract compilation, even though
there are fewer represented states in the reduction relation, that
there is a bisimulation between the two. Particularly, the compiled
semantics is a WEB refinement (defined in pages 57-64 of Manolios'
dissertation) of the non-compiled semantics. We equate states that
``commit'' to non-$\ev{}$ states.

To differentiate the two states spaces, denote the machine
configuration space from the abstractly-compiled machine as
$\CompState$. We additionally denote the reduction relation as $\cmachstep$.

\begin{align*}
\commit(\ev[^\mcntr]{\mexp,\menv,\msto,\mkont}) &= \commitev(\mcntr,\menv,\msto,\mkont,\mexp) \\
\commit(\mstate) &= \mstate \quad \text{otherwise} \\
\commitev(\mcntr,\menv,\msto,\mkont,\svar{\mvar}) &= \co{\mkont,\saddr{\menv(\mvar)},\msto} \\
\commitev(\mcntr,\menv,\msto,\mkont,\slit\mlit) &= \co{\mkont,\slit\mlit,\msto} \\
\commitev(\mcntr,\menv,\msto,\mkont,\slam{\mvar,\mexp,\menv}) &= \co{\mkont,\clos{\mvar,\mexp,\menv},\msto} \\
\commitev(\mcntr,\menv,\msto,\mkont,\sapp[^\mlab]{\mexpi0}{\mexpi1}) &=
  \commitev(\mcntr,\menv,\msto',\kar[^\mcntr_\mlab]{\mexpi1, \menv, \maddr},\mexpi0) \\
 \text{ where } \maddr, \msto' &= \push_\mlab^\mcntr(\msto,\mkont) \\
\commitev(\mcntr,\menv,\msto,\mkont,\sif[^\mlab]{\mexpi0}{\mexpi1}{\mexpi2}) &=
  \commitev(\mcntr,\menv,\msto',\kif[^\mcntr_\mlab]{\mexpi1, \mexpi2, \menv, \maddr},\mexpi0) \\
 \text{ where } \maddr, \msto' &= \push_\mlab^\mcntr(\msto,\mkont)
\end{align*}

Next, the refinement map from non-compiled to compiled states.
\begin{align*}
r &: \State \to \CompState \\
%% STATES
r(\ev[^\mcntr]{\mexp,\menv,\msto,\mkont}) &= r(\commit(\ev[^\mcntr]{\mexp,\menv,\msto,\mkont})) \\
r(\co{\mkont, \mval,\msto}) &= \co{r(\mkont), r(\mval), r(\msto)} \\
r(\ap[_\mlab^\mcntr]{\mvalx{u},\mval,\msto,\mkont}) &= \ap[_\mlab^\mcntr]{r(\mvalx{u}),r(\mval),r(\msto),r(\mkont)} \\
r(\ans{\msto,\mval}) &= \ans{r(\msto), r(\mval)} \\[2mm]
%% STORE
r(\msto) &= \lambda \maddr. \setof{r(\mval) \mid \mval \in \msto(\maddr)} \\[2mm]
%% CONTINUATIONS
r(\kmt) &= \kmt \\
r(\kar[^\mcntr_\mlab]{\mexp,\menv,\maddr}) &= \kar[^\mcntr_\mlab]{\compile{\mexp},\menv,\maddr} \\
r(\kfn[^\mcntr_\mlab]{\maddralt,\maddr}) &= \kfn[^\mcntr_\mlab]{\maddralt,\maddr} \\
r(\kif[^\mcntr]{\mexpi0,\mexpi1,\menv,\maddr}) &= \kif[^\mcntr]{\compile{\mexpi0},\compile{\mexpi1},\menv,\maddr} \\[2mm]
%% VALUES
r(\mop) &= \mop \\
r(\mlit) &= \mlit \\
r(\saddr{\maddr}) &= \saddr{\maddr} \\
r(\clos{\mvar,\mexp,\menv}) &= \clos{\mvar,\compile{\mexp},\menv}
\end{align*}
$r$ terminates because the non-structurally descreasing call case is
guaranteed to not happen unless on a structurally smaller
counterpart. This is because $\commit$ never returns an $\ev{}$
state.

Next we relate states across the different machines with an
equivalence relation $B$ on the two state spaces $S = \State \cup \CompState$,
such that $\forall \mstate \in \State. s B r(s)$. Let $B$ be the reflexive, symmetric closure of $B^*$:
\begin{mathpar}
\inferrule{r(s) = s'}{s B^* s'}
\end{mathpar}

Finally, we show that $B$ is a WEB on the transition system
$\langle S, \Rightarrow \rangle$ where $\Rightarrow = \machstep \cup \cmachstep$.

Let $\langle W, \lessdot \rangle$ be the well-ordered set of expressions (ordered structurally) with a bottom element $\bot$.
\begin{align*}
\erankt(\ev[^\mcntr]{\mexp,\menv,\msto,\mkont}) &= \mexp \\
\erankt(\mstate) &= \bot \quad \text{otherwise} \\
\erankl(s,s') &= 0 \quad \text{{\it Unnecessary}}
\end{align*}

We need one lemma:
\begin{lemma}[Compile/Commit]
For all $\mcntr,\mexp,\menv,\msto,\mkont$,
$\compile{\mexp}^\mcntr(\menv,r(\msto),r(\mkont)) = r(\ev[^\mcntr]{\mexp,\menv,\msto,\mkont})$.
\end{lemma}
\begin{proof}
By induction on $\mexp$.
\begin{byCases}
 \case{\text{Base: } \svar{\mvar}}{By definitions of $\compile{\_}, r, \commit$.}
 \case{\text{Base: } \slit{\mlit}}{By definitions of $\compile{\_}, r, \commit$.}
 \case{\text{Induction step: } \slam{\mvar}{\mexp'}}{By definitions of $\compile{\_}, r, \commit$.}
 \case{\text{Induction step: } \sapp[^\mlab]{\mexpi0}{\mexpi1}}{
   By IH, $\compile{\mexpi0}^\mcntr(\menv,r(\msto'),r(\mkont')) =
           r(\ev[^\mcntr]{\mexpi0}{\menv,\msto',\mkont'})$
   where $\mkont' = \kar[^\mcntr_\mlab]{\mexpi1,\menv,\maddr}$
     and $\maddr, \msto' = \push_\mlab^\mcntr(\msto,\mkont')$.
  Thus holds by definitions of $r, \commit, \compile{\_}$.}
 \case{\text{Induction step: } \sif{\mexpi0}{\mexpi1}{\mexpi2}}{
   By IH, $\compile{\mexpi0}^\mcntr(\menv,r(\msto'),r(\mkont')) =
           r(\ev[^\mcntr]{\mexpi0}{\menv,\msto',\mkont'})$
   where $\mkont' = \kif[^\mcntr_\mlab]{\mexpi1,\mexpi2,\menv,\maddr}$
     and $\maddr, \msto' = \push_\mlab^\mcntr(\msto,\mkont')$.
  Thus holds by definitions of $r, \commit, \compile{\_}$.}
\end{byCases}
\end{proof}

\begin{theorem}
$B$ is a WEB on the transition system $\langle S, \Rightarrow \rangle$.
\end{theorem}
\begin{proof}
Let $s,u,w \in S$ be arbitrary such that $s B w$ and $s \Rightarrow u$.
If $w = s$, the first case of WEB trivially holds with witness $u$. We assume $w \neq s$. Thus $w = r(s)$.
By cases on $s \Rightarrow u$:
\begin{byCases}
%% NON-COMPILED
%% EVAL
\case{
 \ev{\svar\mvar,\menv,\msto,\mkont} \machstep
 \co{\mkont,\saddr{\menv(\mvar)},\msto}}{

Since $w = r(s)$, $w = r(u)$ by definition of $r$. The second case of
WEB holds by definition of $\erankt$, $\lessdot$ and case analysis on
$w$.}

\case{
 \ev{\slit\mlit,\menv,\msto,\mkont} \machstep
 \co{\mkont,\mlit,\msto}}{

Since $w = r(s)$, $w = r(u)$ by definition of $r$. The second case of
WEB holds by definition of $\erankt$, $\lessdot$ and case analysis on
$w$.}

\case{
 \ev{\slam\mvar\mexp,\menv,\msto,\mkont} \machstep
 \co{\mkont,\clos{\mvar,\mexp,\menv},\msto}}{

Since $w = r(s)$, $w = r(u)$ by definition of $r$. The second case of
WEB holds by definition of $\erankt$, $\lessdot$ and case analysis on
$w$.}

\case{
 \ev[^\mcntr]{\sapp[^\mlab]{\mexpi0}{\mexpi1},\menv,\msto,\mkont} \machstep
 \ev[^\mcntr]{\mexpi{0},\menv,\msto',\kar[_\mlab^\mcntr]{\mexpi{1},\menv,\maddr}}}{
  where $\maddr,\msto' = \push^\mcntr_\mlab(\msto,\mkont)$

By definition of $\commit$, $r(u) = w$, thus $u B w$.  By definition
of $\lessdot$, $erankt(u) < erankt(s)$.  Thus the second case of WEB
holds.}

\case{
 \ev[^\mcntr]{\sif[^\mlab]{\mexpi0}{\mexpi1}{\mexpi2},\menv,\msto,\mkont} \machstep
 \ev[^\mcntr]{\mexpi0,\menv,\msto',\kif[^\mcntr]{\mexpi1,\mexpi2,\menv,\maddr}}}{
  where $\maddr,\msto' = \push_\mlab^\mcntr(\msto,\mkont)$

By definition of $\commit$, $r(u) = r(s) = w$. By definition of
$\lessdot$, $\erankt(u) \lessdot \erankt(s)$. Thus the second case of
WEB holds.}

%% CONTINUE
\case{
 \co{\kmt,\mval,\msto} \machstep
 \ans{\msto,\mvalx{u}}}{
  where $\mvalx{u} \in \force(\msto, \mval)$

By definition of $\Rightarrow, \cmachstep$, $w \Rightarrow r(u)$, satisfying the
first case of WEB.}

\case{
 \co{\kar[^\mcntr_\mlab]{\mexp,\menv,\maddr},\mval,\msto} \machstep
 \ev[^\mcntr]{\mexp,\menv,\msto',\kfn[^\mcntr_\mlab]{\maddralt,\maddr}}}{
  where $\maddralt,\msto' = \stash_\mlab^\mcntr(\msto,\mval)$

By definition of $\Rightarrow, \cmachstep$,
$w \Rightarrow \compile{\mexp}^{\mcntr}(\menv,r(\msto'),r(\kfn[^\mcntr_\mlab]{\maddralt,\maddr}))$.
By the compile/commit lemma, $w \Rightarrow r(u)$.
Thus the first case of WEB holds with witness $r(u)$.}

\case{
 \co{\kfn[^\mcntr_\mlab]{\maddralt,\maddr},\mval,\msto}  \machstep
 \ap[^\mcntr_\mlab]{\mvalx{u},\mval,\mkont,\msto}}{
  where $\mkont \in \msto(\maddr), \mvalx{u} \in \msto(\maddralt)$

By definition of $\cmachstep$, $w \cmachstep r(u)$, satisfying the
first case of WEB.}

\case{
 \co{\kif[^\mcntr]{\mexpi0,\mexpi1,\menv,\maddr},\strue,\msto}  \machstep
 \ev[^\mcntr]{\mexpi0,\menv,\msto,\mkont}}{
  where $\mkont\in\msto(\maddr)$

By definition of $\Rightarrow, \cmachstep$, $w \Rightarrow \compile{\mexpi0}^{\mcntr}(\menv,r(\msto),r(\mkont))$.
By the compile/commit lemma, $w \Rightarrow r(u)$.
Thus the first case of WEB holds with witness $r(u)$.}

\case{
 \co{\kif[^\mcntr]{\mexpi0,\mexpi1,\menv,\maddr},\sfalse,\msto}  \machstep
 \ev[^\mcntr]{\mexpi1,\menv,\msto,\mkont}}{
  where $\mkont\in\msto(\maddr)$

By definition of $\Rightarrow, \cmachstep$, $w \Rightarrow \compile{\mexpi0}^{\mcntr}(\menv,r(\msto),r(\mkont))$.
By the compile/commit lemma, $w \Rightarrow r(u)$.
Thus the first case of WEB holds with witness $r(u)$.}

%% APPLY
\case{
 \ap[^\mcntr_\mlab]{\clos{\mvar,\mexp,\menv},\mval,\msto,\mkont}  \machstep
 \ev[^{\mcntr'}\!]{\mexp,\menv',\msto',\mkont}}{
  where $\menv',\msto',\mcntr' = \bind^{ \mcntr}_\mlab(\menv,\msto,\mvar,\mval)$

By definition of $\Rightarrow, \cmachstep$,
$w \Rightarrow \compile{\mexp}^{\mcntr'}(\menv',r(\msto'),r(\mkont))$.
By the compile/commit lemma, $w \Rightarrow r(u)$.
Thus the first case of WEB holds with witness $r(u)$.}

\case{
 \ap[^\mcntr_\mlab]{\mop,\mval,\msto,\mkont}  \machstep
 \co{\mkont,\mval',\msto}}{
  where $\mvalx{u} \in \force(\msto, \mval), \mval'\in\interpdelta(\mop,\mvalx{u})$

By definition of $\cmachstep$, $w \cmachstep r(u)$, satisfying the
first case of WEB.}

%% COMPILED

\case{s \cmachstep u}{Must be the case that $s = w$, thus the first case of WEB holds.}
\end{byCases}
% (a) case is non-compiled to non-ev
% (b) is for s \equiv ev w is r(s)
% (c) is for s compiled, r(w) = s
\end{proof}

\subsection{Soundness of widened abstract compilation}

\begin{align*}
\prep(S,F,\masto) = S \cup \setof{\ttuple{\mastate}{\masto} \mid \mastate \in F}
\end{align*}

\begin{mathpar}
\inferrule{\masto \sqsubseteq \masto'}{(\mastate,\masto) \sqsubseteq (\mastate,\masto')} \qquad
\inferrule{\prep(S,F,\masto) \sqsubseteq \prep(S',F',\masto')}
          {(S,F,\masto) \sqsubseteq (S',F',\masto')}
\end{mathpar}

\begin{align*}
\alpha(\mstate) &= (\setof{(\mastate,\masto)},\setof{\mastate},\masto) \\
 \text{ where } \mastate,\masto &= \nw(\mstate) \\
\gamma((S,F,\masto)) &=
  \setof{\wn(\mastate,\msto) \mid
         \ttuple{\mastate}{\masto} \in \prep(S,F,\masto),
         \masto \in \gamma(\masto)} \\
% set of all functions that for all \maddr \in \dom(\masto), map \maddr to a subset of \masto(\maddr)
\gamma(\masto) &= \setof{R' \mid R' \subseteq R, R' \text{ functional}, \dom(R') = \dom(\masto)} \\
 \text{ where } R &= \setof{\ttuple{\maddr}{\hat{\mstor}} \mid
                            \maddr \in\dom(\masto),
                            \hat{\mstor} \subseteq \masto(\maddr)} \\
\alpha^*(C) &= \setof{\alpha(\mstate) \mid \mstate \in C} \\
\gamma^*(A) &= \bigcup\limits_{\mastate \in A}{\gamma(\mastate)}
\end{align*}

\begin{lemma}
$\gamma^* \circ \alpha^* \ge 1_C$
\end{lemma}
\begin{proof}
This is immediate if $\masto \in \gamma(\masto)$, which is true, since
$\masto(\maddr) \subseteq \masto(\maddr)$.
\end{proof}

\begin{lemma}
$\alpha^* \circ \gamma^* \le 1_C$
\end{lemma}
\begin{proof}
TODO
\end{proof}

\begin{theorem}
If $\mstate \cmachstep \mstate'$ and $\alpha(\mstate) \sqsubseteq (S,F,\masto)$
then there exist $S',F',\masto'$ such that
$(S,F,\masto) \machstep (S',F',\masto')$ and
$\alpha(\mstate') \sqsubseteq (S',F',\masto')$
\end{theorem}
\begin{proof}
By definition of $\alpha, \sqsubseteq$, there exists a $\mastate \in F$
such that, with $\mastate^*,\masto^* = \nw(\mstate)$,
$\mastate = \mastate^*$ and $\masto^* \sqsubseteq \masto$.

Thus, by definition of $\machstep, \nw, \wn,\prep,\sqsubseteq$ (call the result $(S',F',\masto')$),
letting $\mastate^*_1,\masto^*_1 = \nw(\mstate')$,
$\masto^*_1 \sqsubseteq \masto'$ and
$\setof{\ttuple{\mastate^*_1}{\masto'}} \sqsubseteq \prep(S',F',\masto')$.
Thus $\alpha(\mstate') \sqsubseteq(S', F', \masto')$.
\end{proof}

\subsection{Semantic equivalence with locally log-based store deltas}

Here we show extensional equality of the relations. We will use $\dmachstep$ for the store
delta semantics and $\dcompile{\_}$ for its compilation function.

Let $\equiv \subseteq \Expr$ be the reflexive, transitive,
symmetric closure of $\equiv^*$ with structural lifting where non-$\Expr$ elements are compared with equality, and that lifted to functions $\Addr \to \wp(\Storeable)$.
\begin{mathpar}
\inferrule{ }{\dcompile{\mexp} \equiv^* \compile{\mexp}}
\end{mathpar}

\begin{lemma}
For all $\mexp, \mcntr,\menv,\msto,\msdiff,\mkont,\msto^*,\mkont^*$,
if $\msto \equiv \msto^*$ and $\mkont \equiv \mkont^*$, then
$\dcompile{\mexp}^\mcntr(\menv,\msto,\msdiff,\mkont) = \ttuple{\mastate}{\msdiff'}$ iff
$\compile{\mexp}^\mcntr(\menv,\msto^*,\mkont^*) = \mstate$ such that
$\mastate,\replay(\msdiff',\msto) \equiv \nw(\mstate)$
\end{lemma}
\begin{proof}
By induction on $\mexp$
\begin{byCases}
 \case{\text{Base } \svar{\mvar}}{By definitions of $\dcompile{\_},\compile{\_},\replay,\nw$.}
 \case{\text{Base } \slit{\mlit}}{By definitions of $\dcompile{\_},\compile{\_},\replay,\nw$.}
 \case{\text{Induction step } \slam{\mvar}{\mexp}}{By definitions of $\dcompile{\_},\compile{\_},\replay,\nw$.}
 \case{\text{Induction step } \sapp{\mexpi0}{\mexpi1}}{
  By IH and definitions of $\push, \equiv,\replay$,
   $\dcompile{\mexpi0}^\mcntr(\menv,\msto,\msdiff',\kar[^\mcntr_\mlab]{\dcompile{\mexpi1},\menv,\maddr})$
    where $\maddr,\msdiff' = \push_\mlab^\mcntr(\msdiff,\mkont)$ iff
   $\compile{\mexpi0}^\mcntr(\menv,\msto^*_1,\kar[^\mcntr_\mlab]{\compile{\mexpi1},\menv,\maddr})$
    where $\maddr,\msto^*_1 = \push_\mlab^\mcntr(\msto^*,\mkont^*)$.
   Thus the conclusion holds by definitions of $\dcompile{\_}, \compile{\_}$.}
 \case{\text{Induction step } \sif{\mexpi0}{\mexpi1}{\mexpi2}}{
  By IH and definitions of $\push, \equiv,\replay$,
   $\dcompile{\mexpi0}^\mcntr(\menv,\msto,\msdiff',\kif[^\mcntr_\mlab]{\dcompile{\mexpi1},\dcompile{\mexpi2},\menv,\maddr})$
    where $\maddr,\msdiff' = \push_\mlab^\mcntr(\msdiff,\mkont)$ iff
   $\compile{\mexpi0}^\mcntr(\menv,\msto^*_1,\kif[^\mcntr_\mlab]{\compile{\mexpi1},\compile{\mexpi2},\menv,\maddr})$
    where $\maddr,\msto^*_1 = \push_\mlab^\mcntr(\msto^*,\mkont^*)$.
   Thus the conclusion holds by definitions of $\dcompile{\_}, \compile{\_}, \nw$.}
\end{byCases}
\end{proof}

\begin{theorem}
%For all $S, F, \msto, S', F', \msto', S^*, F^*, \msto^*$,
If $S \equiv S^*, F \equiv F^*, \msto \equiv \msto^*$, then
$(S,F,\msto) \machstep (S',F',\msto')$ iff
$\exists S^*_1, F^*_1, \msto^*_1.
  S' \equiv S^*_1 \wedge F' \equiv F^*_1 \wedge \msto' \equiv \msto^*_1 \wedge
  (S^*,F^*,\msto^*) \dmachstep (S^*_1,F^*_1,\msto^*_1)$
\end{theorem}
\begin{proof}
By definitions of $\machstep, \dmachstep, \push, \stash, \replay,
\replayall$, commutativity and associativity of $\sqcup$, and the previous lemma.
\end{proof}

\subsection{Semantic equivalence of log-based updates to a timestamped store}

Because the store is monotonically increasing, we know that
$\sqsubseteq$ forms a total order on stores in the system. We use this
information to sort and index the stores. Call the timestamped reduction relation $\nmachstep$.

\begin{align*}
\alpha((S,F,\msto)) &= (\alpha(S,\Sigma), F, \Sigma, |\Sigma|-1) \\
\text{ where } \Sigma &= \lambda n. \sorted_n \\
               \sorted &= \sort(\setof{\msto \mid \ttuple{\_}{\msto} \in \prep(S,F,\msto)},\sqsubseteq) \\
\alpha(S,\Sigma) &=
   \setof{\ttuple{\mstate}
                 {\map(\Sigma,\sort(\setof{\msto \mid \ttuple{\mstate}{\msto} \in S}, \sqsupseteq))}
           \mid \ttuple{\mstate,\_} \in S} \\
\gamma((S,F,\Sigma,n)) &= (\gamma(S,\Sigma), F, \Sigma(n)) \\
\gamma(S,\Sigma) &= \setof{\ttuple{\mstate}{\Sigma(\ell_i)} \mid S(\mstate) = \ell, 0 \le i < |\ell|}
\end{align*}

%% \begin{lemma}
%% $\gamma \circ \alpha = 1_C$
%% \end{lemma}
%% \begin{proof}
%% TODO
%% \end{proof}

%% \begin{lemma}
%% $\alpha \circ \gamma \circ = 1_C$
%% \end{lemma}
%% \begin{proof}
%% TODO
%% \end{proof}

\begin{lemma}
For all $\msdiff \text{ finite},\msto,\joinp$, let $\replaychange(\msdiff,\msto,\joinp) = \msto',\joinp'$.
$\joinp \vee (\joinp' \iff \msto \neq \msto')$
\end{lemma}
\begin{proof}
By induction on $\msdiff$.
\begin{byCases}
 \case{\text{Base } \bot}{By definition of $\replaychange$, $\msto = \msto'$ and $\joinp = \joinp'$.}
 \case{\text{Induction step } \msdiff'[\maddr \mapsto \hat{\mstor}]}{
  Let $\replaychange(\msdiff, \msto^*, \joinp \vee \joinp^*) = \msto^*_1,\joinp^*_1$
   where $\hat{\mstor}' = \hat{\mstor} \sqcup \msto(\maddr)$, $\msto^* = \msto[\maddr \mapsto \hat{\mstor}']$, $\joinp^* = \hat{\mstor}' \deceq \msto(\maddr)$.
  If $\joinp^*$, then $\msto^*_1 \neq \msto$ and $\joinp^*_1$ because $\replaychange$ monotonically increases $\msto$ and $\joinp$.
  Otherwise, by IH, if $\joinp$, then $\joinp^*_1$; otherwise, $\msto^*_1 \neq \msto^* \iff \joinp^*_1$}
\end{byCases}
\end{proof}

\begin{lemma}
For all $\Xi \text{ finite}, \msto, \updatedp$, let $\replaychangeall(\Xi,\msto,\updatedp) = \msto',\updatedp'$.
$\updatedp \vee (\updatedp' \iff \msto \neq \msto')$
\end{lemma}
\begin{proof}
By induction on $\Xi$.
\begin{byCases}
 \case{\text{Base } \varnothing}{By definition of $\replaychange$, $\msto = \msto'$ and $\updatedp = \updatedp'$.}
 \case{\text{Induction step } \setof{\msdiff}\cup\Xi'}{
  Let $\replaychangeall(\Xi, \msto^*, \updatedp \vee \updatedp^*) = \msto^*_1,\updatedp^*_1$
   where $\replaychange(\msdiff,\msto,\updatedp) = \msto^*,\updatedp^*$.
   By the previous lemma, $\updatedp \vee (\updatedp^* \iff \msto \neq \msto^*)$.
   If $\updatedp^*$ then $\msto \neq msto^*_1$ and $\updatedp^*_1$ because $\replaychangeall$ monotonically increases $\msto$ and $\updatedp$.
   Otherwise, by IH, if $\updatedp$, then $\updatedp^*_1$; otherwise $\msto^*_1 \neq \msto^* \iff \updatedp^*_1$}
\end{byCases}
\end{proof}

\begin{theorem}
If $(S,F,\msto) \machstep (S',F',\msto')$ and $\alpha((S,F,\msto))
\sqsubseteq (S^*,F,\Sigma,n)$ then there exist $S^*_1,\Sigma',n'$ such
that $(S^*,F,\Sigma,n) \nmachstep (S^*_1,F',\Sigma',n')$ and
$\alpha((S',F',\msto')) \sqsubseteq (S^*_1,F',\Sigma',n')$.
\end{theorem}
\begin{proof}
TODO
\end{proof}

\begin{theorem}
If $(S,F,\Sigma,n) \machstep (S',F',\Sigma',n')$ and
$\gamma((S,F,\Sigma,n)) \sqsubseteq (S^*,F,\msto)$ then there exist
$S^*_1,\msto'$ such that $(S^*,F,\msto) \machstep (S^*_1,F',\msto')$
and $\gamma((S',F',\Sigma',n')) \sqsubseteq (S^*_1,F',\msto')$.
\end{theorem}
\begin{proof}
TODO
\end{proof}

\end{document}
